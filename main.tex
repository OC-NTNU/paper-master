%% BioMed_Central_Tex_Template_v1.06
%%                                      %
%  bmc_article.tex            ver: 1.06 %
%                                       %

%%IMPORTANT: do not delete the first line of this template
%%It must be present to enable the BMC Submission system to
%%recognise this template!!

%%%%%%%%%%%%%%%%%%%%%%%%%%%%%%%%%%%%%%%%%
%%                                     %%
%%  LaTeX template for BioMed Central  %%
%%     journal article submissions     %%
%%                                     %%
%%          <8 June 2012>              %%
%%                                     %%
%%                                     %%
%%%%%%%%%%%%%%%%%%%%%%%%%%%%%%%%%%%%%%%%%


%%%%%%%%%%%%%%%%%%%%%%%%%%%%%%%%%%%%%%%%%%%%%%%%%%%%%%%%%%%%%%%%%%%%%
%%                                                                 %%
%% For instructions on how to fill out this Tex template           %%
%% document please refer to Readme.html and the instructions for   %%
%% authors page on the biomed central website                      %%
%% http://www.biomedcentral.com/info/authors/                      %%
%%                                                                 %%
%% Please do not use \input{...} to include other tex files.       %%
%% Submit your LaTeX manuscript as one .tex document.              %%
%%                                                                 %%
%% All additional figures and files should be attached             %%
%% separately and not embedded in the \TeX\ document itself.       %%
%%                                                                 %%
%% BioMed Central currently use the MikTex distribution of         %%
%% TeX for Windows) of TeX and LaTeX.  This is available from      %%
%% http://www.miktex.org                                           %%
%%                                                                 %%
%%%%%%%%%%%%%%%%%%%%%%%%%%%%%%%%%%%%%%%%%%%%%%%%%%%%%%%%%%%%%%%%%%%%%

%%% additional documentclass options:
%  [doublespacing]
%  [linenumbers]   - put the line numbers on margins

%%% loading packages, author definitions

%\documentclass[twocolumn]{bmcart}% uncomment this for twocolumn layout and comment line below
\documentclass{bmcart}

%%% Load packages
%\usepackage{amsthm,amsmath}
%\RequirePackage{natbib}
%\RequirePackage{hyperref}
\usepackage[utf8]{inputenc} %unicode support
%\usepackage[applemac]{inputenc} %applemac support if unicode package fails
%\usepackage[latin1]{inputenc} %UNIX support if unicode package fails

\usepackage{natbib}
\usepackage{hyperref}
\usepackage{graphicx}
\usepackage{multirow}
\usepackage{rotating}
\usepackage{xcolor,colortbl}
\usepackage[color=red]{todonotes}
\usepackage{qtree}
\usepackage{tikz-dependency}


%%%%%%%%%%%%%%%%%%%%%%%%%%%%%%%%%%%%%%%%%%%%%%%%%
%%                                             %%
%%  If you wish to display your graphics for   %%
%%  your own use using includegraphic or       %%
%%  includegraphics, then comment out the      %%
%%  following two lines of code.               %%
%%  NB: These line *must* be included when     %%
%%  submitting to BMC.                         %%
%%  All figure files must be submitted as      %%
%%  separate graphics through the BMC          %%
%%  submission process, not included in the    %%
%%  submitted article.                         %%
%%                                             %%
%%%%%%%%%%%%%%%%%%%%%%%%%%%%%%%%%%%%%%%%%%%%%%%%%


\def\includegraphic{}
\def\includegraphics{}



%%% Put your definitions there:
\startlocaldefs
\endlocaldefs


%%% Begin ...
\begin{document}

%%% Start of article front matter
\begin{frontmatter}

\begin{fmbox}
\dochead{Research}

%%%%%%%%%%%%%%%%%%%%%%%%%%%%%%%%%%%%%%%%%%%%%%
%%                                          %%
%% Enter the title of your article here     %%
%%                                          %%
%%%%%%%%%%%%%%%%%%%%%%%%%%%%%%%%%%%%%%%%%%%%%%

\title{Paper on Information Extraction and such}

%%%%%%%%%%%%%%%%%%%%%%%%%%%%%%%%%%%%%%%%%%%%%%
%%                                          %%
%% Enter the authors here                   %%
%%                                          %%
%% Specify information, if available,       %%
%% in the form:                             %%
%%   <key>={<id1>,<id2>}                    %%
%%   <key>=                                 %%
%% Comment or delete the keys which are     %%
%% not used. Repeat \author command as much %%
%% as required.                             %%
%%                                          %%
%%%%%%%%%%%%%%%%%%%%%%%%%%%%%%%%%%%%%%%%%%%%%%

\author[
   addressref={aff1},                   % id's of addresses, e.g. {aff1,aff2}
   corref={aff1},                       % id of corresponding address, if any
   noteref={n1},                        % id's of article notes, if any
   email={elias.aamot@gmail.com}   % email address
]{\inits{EAA}\fnm{Elias} \snm{Aamot}}
\author[
   addressref={aff1,aff2},
   email={john.RS.Smith@cambridge.co.uk}
]{\inits{XY}\fnm{X} \snm{Y}}

%%%%%%%%%%%%%%%%%%%%%%%%%%%%%%%%%%%%%%%%%%%%%%
%%                                          %%
%% Enter the authors' addresses here        %%
%%                                          %%
%% Repeat \address commands as much as      %%
%% required.                                %%
%%                                          %%
%%%%%%%%%%%%%%%%%%%%%%%%%%%%%%%%%%%%%%%%%%%%%%

\address[id=aff1]{%                           % unique id
  \orgname{Department of Zoology, Cambridge}, % university, etc
  \street{Waterloo Road},                     %
  %\postcode{}                                % post or zip code
  \city{London},                              % city
  \cny{UK}                                    % country
}
\address[id=aff2]{%
  \orgname{Marine Ecology Department, Institute of Marine Sciences Kiel},
  \street{D\"{u}sternbrooker Weg 20},
  \postcode{24105}
  \city{Kiel},
  \cny{Germany}
}

%%%%%%%%%%%%%%%%%%%%%%%%%%%%%%%%%%%%%%%%%%%%%%
%%                                          %%
%% Enter short notes here                   %%
%%                                          %%
%% Short notes will be after addresses      %%
%% on first page.                           %%
%%                                          %%
%%%%%%%%%%%%%%%%%%%%%%%%%%%%%%%%%%%%%%%%%%%%%%

\begin{artnotes}
%\note{Sample of title note}     % note to the article
\note[id=n1]{Equal contributor} % note, connected to author
\end{artnotes}

\end{fmbox}% comment this for two column layout

%%%%%%%%%%%%%%%%%%%%%%%%%%%%%%%%%%%%%%%%%%%%%%
%%                                          %%
%% The Abstract begins here                 %%
%%                                          %%
%% Please refer to the Instructions for     %%
%% authors on http://www.biomedcentral.com  %%
%% and include the section headings         %%
%% accordingly for your article type.       %%
%%                                          %%
%%%%%%%%%%%%%%%%%%%%%%%%%%%%%%%%%%%%%%%%%%%%%%

\begin{abstractbox}

\begin{abstract} % abstract
\parttitle{First part title} %if any
Text for this section.

\parttitle{Second part title} %if any
Text for this section.
\end{abstract}

%%%%%%%%%%%%%%%%%%%%%%%%%%%%%%%%%%%%%%%%%%%%%%
%%                                          %%
%% The keywords begin here                  %%
%%                                          %%
%% Put each keyword in separate \kwd{}.     %%
%%                                          %%
%%%%%%%%%%%%%%%%%%%%%%%%%%%%%%%%%%%%%%%%%%%%%%

\begin{keyword}
\kwd{sample}
\kwd{article}
\kwd{author}
\end{keyword}

% MSC classifications codes, if any
%\begin{keyword}[class=AMS]
%\kwd[Primary ]{}
%\kwd{}
%\kwd[; secondary ]{}
%\end{keyword}

\end{abstractbox}
%
%\end{fmbox}% uncomment this for twcolumn layout

\end{frontmatter}


%%%%%%%%%%%%%%%%%%%%%%%%%%%%%%%%%%%%%%%%%%%%%%
%% Main stuff
%%%%%%%%%%%%%%%%%%%%%%%%%%%%%%%%%%%%%%%%%%%%%%
\section{Introduction}

\todo[inline]{introduction, domain, long term goal, context of paper, corpus}

\subsection{Extraction task}

\todo[inline]{We are extracting on a graph based fashion, in order to build structures}

\todo{How is the level of detail here regarding the annotation scheme?}

Of interest in the domain is the extraction of changes to quantitative variables, such as \emph{increase in atmospheric CO$_2$}, and interactions between such changes, which can be either causal or correlative relations. Table \ref{ann_scheme} presents the trigger categories that were used to annotate text spans of interest in the corpus.

\begin{figure}
\Tree[.CATEGORY [.ENTITY \textit{Variable} \textit{Thing} ]
          [.EVENT [.\textit{Change} \textit{Increase} \textit{Decrease} ] 
         		  [.INTERACTION \textit{Cause} \textit{Correlate} ] 
          ] 
     ]
\caption{Type Hierarchy for the Annotation Scheme}
\label{ann_scheme}
\end{figure}

\emph{Variable} in our annotation scheme is defined as a \emph{quantitative variable}, meaning an entity than can be measured and assigned some value along some ordered axis, either as a numerical value or a value from a totally ordered set of discrete states. This includes among other counts, frequencies and ratios. Text spans that can be interpreted as quantitative variables out of context are abundant in the scientific literature, but only a small subset of these are of interest. Only variables that occur in an event are therefore annotated in the corpus. 

\emph{Thing} is used to annotate any span of text that functions as the argument of a change event, while not fulfilling the requirements to be annotated as a variable. 

A change event describes a directional, quantitative change in the value of a quantitative variable. \emph{Increase} is used to annotate a change in the positive direction, \emph{decrease} annotates changes in the negative direction and \emph{change} is used when the direction is underspecified in the text. All the change events take an entity as the \emph{theme} argument, signalling the variable that undergoes the change.

Interaction categories mark text spans that explicitly signal interactions between two change events. The types of interactions that are used in the corpus are \emph{cause} and \emph{correlate}. Cause events take two arguments: An \emph{agent}, which marks the change event that causes the other change event to come about. The \emph{theme} argument is used for the caused change event. Correlation events also take two arguments, the \emph{theme} and \emph{co-theme}. It was noticed during corpus annotation that there is a tendency in natural language to focus on one of the changes as initiating the other, giving such correlations a directional interpretation. For correlations with an directional interpretation, the initiating event is marked as the theme, and the other event is marked as the co-theme. If the correlation has no directional interpretation, the event that is the syntactic object of the correlation trigger is marked as co-theme, and the other event as theme by default.

Three additional categories are used in the corpus to handle some common language constructs, namely conjunction, disjunction and referring expressions. The categories \emph{and} and \emph{or} handle conjunctions and disjunctions, respectively, and each take two \emph{part} arguments. Referring expressions are handled by the \emph{RefExp} categories, which takes the antecedent as the \emph{coref} argument.

For a more detailed introduction to the annotation scheme used in the corpus, see \citet{mar14}.

\subsection{Event extraction approaches}

Event extraction is traditionally conducted with a bottom-up approach, in which entities are first extracted, and subsequently used as evidence during a second step in which events are extracted. However, it is also possible to conceive a top-down approach, where events are extracted first, and the second step extracts entities. 

The bottom-up approach is likely common in information extraction because in most extraction tasks, entity extraction is a separate, well-defined task that is often considered easier than event extraction. However, in the extraction task adapted here, it is impossible to clearly separate entity and event extraction, as text spans are only considered entities if they occur as the argument of an event.

It is therefore of interest to see whether the top-down approach is better suited for the task at hand than the bottom-up approach.
\section{Methods}

An evaluation was conducted to compare the performance of the two information extraction approaches on the specified task. Statistical information extraction systems traditionally follow the bottom-up approach, and an existing information exaction system was therefore selected to provide the benchmark for the bottom-up approach. On the other hand, the top-down approach is rarely taken in the literature. It was therefore necessary to develop a prototype top-down information extraction system for the sake of comparison.

\subsection{Bottom-up system: TEES}

To act as a benchmark bottom-up system, the Turku Event Extraction System (TEES) \citep{bjö11ddi} was selected. TEES was selected for the evaluation because it is has shown state-of-the-art performance over several years of development, can extract events with arbitrary structures, can be trained with custom data sets and the source code is publicly available\footnote{\url{https://github.com/jbjorne/TEES}}. TEES has been developed for BioNLP, and has attained high scores on a number of shared tasks, including BioNLP 2009 (1st place) \citep{bjö09}, BioNLP 2011 (1st place in 4/8 tasks) \citep{bjö11} and Drug-Drug Interactions 2011 Challenge (4th place) \citep{bjö11ddi}.

The TEES pipeline consists of the following steps: Preprocessing, Named Entity Recognition, Trigger Detection, Edge Detection and Unmerging. 

Preprocessing consists of sentence splitting, constituency parsing and conversion to Stanford dependency representation. Sentence splitting is normally conducted using the GENIA sentence splitter\footnote{\url{http://www.nactem.ac.uk/y-matsu/geniass/}}, and parsing normally using the BLLIP-parser \citep{cha05} with the McClosky model for biomedical text \citep{mcc08}. Conversion to dependency format is normally performed by the Stanford dependency converter \citep{dem08}. Because the textual materials in the experiment conducted here were not of the biomedical domain, the Stanford parser \citep{kle03} was used instead for preprocessing.

The Named Entity Recognition, Trigger Detection, Edge Detection and Unmerging components are all machine learning-based multi-class classification components that use the $SVM^{MULTICLASS}$ implementation of SVM for classification.

TEES has participated mostly in event extraction-focused shared tasks, in which named entities were given as input to the system, so the Named Entity Component is normally not used, but can be included into the pipeline if required by the task, which is the case in the experiment presented here. The Named Entity Component makes a linear pass over all the tokens in the sentence, classifying every token as belonging to either one of the entity categories or as not an entity. Normally each single token is counted as an independent entity, but if two or more subsequent tokens are classified as belonging to the same class, they are merged to form a multi-token entity if a multi-token entity with the exact same text string can be found in the training data.

Trigger Detection makes another linear pass over the tokens in the sentence, detecting event trigger words. Every token is either classified as belonging to one of the event categories, or as not an event trigger. In the same way as with NER, multiple subsequent tokens can be merged to a single trigger if given the same class, and a trigger with the exact same text string can be found in the training data.

Edge Detection detects arguments of the triggered events. Every event is paired with every other event and entity, and the classifier determines which argument relation type holds between the pair, if any. Because this scheme lets events take not only entities, but also events as argument, this can create events with any structure desired.

In case of overlapping events, i.e. events that share some trigger, some cleaning up is required to separate the events. This is performed by the Unmerging step, which uses a classifier to determine whether an argument branch should be unmerged into a separate event. 

\subsection{Top-down system: Pattern matching}

In order to benchmark the top-down approach, an information extraction system had to be developed that followed this approach. Inspection of the data revealed that trigger words for events were relatively unambiguous. It was therefore concluded that a deterministic pattern matching system would be sufficient to provide an estimate for the performance of the top-down approach on the task at hand.

In the system, a pattern is defined as a partial dependency graph that signals the presence of an event. During information extraction, the system parses the text with the Stanford parser \citep{kle03}, and an event is detected for every pattern that matches the dependency parse tree of the sentence. In case of a match, each pattern specifies which node(s) in the dependency tree should be labelled as the event trigger, and which parts of the dependency tree that should be labelled as the arguments. 

A set of patterns was manually developed from eight of the papers in the corpus, with the remaining two held out as test data. For every annotated event in the corpus, a pattern was written that would extract that event. To improve performance, each pattern was manually evaluated against the eight development papers, and patterns that yielded a high ratio of false positives on the development data were excluded.

As an illustrative example, Figure \ref{dep_pattern} presents how the pattern \emph{T prep "in" pobj S} with $T=increase$ is matched against the sentences "This reveals a significant increase in jelly depositions." The successful match produces the annotation "This reveals a significant [increase$_{increase}$] in [jelly depositions$_{variable}$]."

\begin{figure}
\begin{center}
\begin{dependency}[theme = simple]
	\begin{deptext}
	This \& reveals \& a \& significant \& \textcolor{blue}{increase} \& \textcolor{red}{in} \& \textcolor{brown}{jelly} \& \textcolor{brown}{depositions} \& . \\
	\end{deptext}
	\depedge{2}{1}{NSUBJ}
	\depedge[arc angle=60]{5}{3}{DET}
	\depedge[arc angle=25]{5}{4}{AMOD}
	\depedge{2}{5}{DOBJ}
	\depedge[edge style=red]{5}{6}{PREP}
	\depedge[arc angle=25, edge style=brown]{8}{7}{NN}
	\depedge[edge style=red]{6}{8}{POBJ}
\end{dependency}
\end{center}
\caption{Matching against the dependency parse tree.}
\label{dep_pattern}
\end{figure}

\subsection{Why do we do it this way?}

\todo[inline]{I need some help regarding where to place this stuff and what to write}


\subsection{Evaluation methods}

TEES was evaluated by five-fold cross-validation, each fold consisting of two papers. As eight papers had been used during development of the pattern matching system, the pattern matching system was evaluated once only, using the two remaining papers as test data.

In the evaluation, an element is matched to an element in the gold standard if the text spans exhibit any degree of overlap, thus abstracting away from exact scoping of text spans. A predicted trigger was counted as a true positive if it matched an element of the same annotation category in the gold standard, and a false positive if it didn't. A trigger in the gold standard was counted as a false negative if it could not be matched to a predicted trigger of the same category. 
\section{Results}

%Table \ref{tees_trigger_conf} displays gold standard versus predicted category for triggers, as predicted by the bottom-up system (TEES), whereas table \ref{pms_trigger_conf} shows the same for the top-down system (PMS). Table \ref{tees_argument_conf} displays gold standard versus predicted category for argument types, as predicted by the bottom-up system (TEES), and table \ref{pms_argument_conf} by the top-down system (PMS).

\iffalse
\begin{sidewaystable}
\begin{center}
\begin{tabular}{ | r | l | r | r | r | r | r | r | r | r | r | r | r |}
	\hline
	\multirow{11}{*}{True} & \multicolumn{12}{c|}{Predicted} \\ \hline
	&\cellcolor{gray} & None & Variable & Thing & Increase & Decrease & Change & Cause & Correlate & And & Or & Sum\\ \cline{2-13}
	&None & \cellcolor{g} 0 & 659 & 53 & 68 & 30 & 41 & 9 & 3 & 4 & 0 & 867 \\
	&Variable & 715 & \cellcolor{g} 494 & 9 & 7 & 3 & 2 & 2 & 2 & 0 & 0 & 1234 \\
	&Thing & 173 & 24 & \cellcolor{g} 33 & 0 & 1 & 1 & 1 & 0 & 0 & 0 & 233 \\
	&Increase & 269 & 18 & 0 & \cellcolor{g} 253 & 5 & 1 & 2 & 1 & 0 & 0 & 549 \\
	&Decrease & 187 & 4 & 0 & 1 & \cellcolor{g} 123 & 0 & 0 & 1 & 0 & 0 & 316 \\
	&Change & 229 & 4 & 2 & 2 & 0 & \cellcolor{g} 44 & 0 & 1 & 0 & 0 & 282 \\
	&Cause & 163 & 3 & 1 & 3 & 0 & 4 & \cellcolor{g} 2 & 0 & 0 & 0 & 176 \\
	&Correlate & 108 & 0 & 1 & 6 & 0 & 0 & 2 & \cellcolor{g} 17 & 0 & 0 & 134 \\ 
	&And & 150 & 0 & 0 & 0 & 0 & 0 & 0 & 0 & \cellcolor{g} 4 & 0 & 154\\ 
	&Or & 10 & 0 & 0 & 0 & 0 & 0 & 0 & 0 & 0 & \cellcolor{g} 0 & 10 \\ \hline
	&Sum& 2004 & 1206 & 99 & 340 & 162 & 93 & 18 & 25 & 8 & 0 & \\ \hline
\end{tabular}
\end{center}
\caption{Confusion matrix for trigger categories, bottom-up system.}
\label{tees_trigger_conf}
\end{sidewaystable}

\begin{sidewaystable}
\begin{center}
\begin{tabular}{ | r | l | r | r | r | r | r | r | r | r | r | r | r |}
	\hline
	\multirow{11}{*}{True} & \multicolumn{12}{c|}{Predicted} \\ \hline
	&\cellcolor{gray} & None & Variable & Thing & Increase & Decrease & Change & Cause & Correlate & And & Or & Sum\\ \cline{2-13}
	&None & \cellcolor{g} 0 & 67 & 5 & 4 & 6 & 18 & 5 & 0 & 0 & 0 & 105 \\
	&Variable & 197 & \cellcolor{g} 209 & 1 & 0 & 6 & 0 & 0 & 0 & 0 & 0 & 413 \\
	&Thing & 19 & 2 & \cellcolor{g} 6 & 0 & 0 & 0 & 0 & 0 & 0 & 0 & 27 \\
	&Increase & 33 & 2 & 0 & \cellcolor{g} 122 & 0 & 0 & 0 & 0 & 0 & 0 & 157 \\
	&Decrease & 32 & 6 & 0 & 0 & \cellcolor{g} 92 & 0 & 0 & 0 & 0 & 0 & 130 \\
	&Change & 30 & 2 & 0 & 0 & 0 & \cellcolor{g} 56 & 0 & 0 & 0 & 0 & 88 \\
	&Cause & 22 & 1 & 0 & 0 & 0 & 4 & \cellcolor{g} 13 & 0 & 0 & 0 & 40 \\
	&Correlate & 46 & 2 & 0 & 0 & 0 & 0 & 0 & \cellcolor{g} 2 & 0 & 0 & 134 \\ 
	&And & 37 & 4 & 0 & 0 & 0 & 0 & 0 & 0 & \cellcolor{g} 0 & 0 & 41 \\ 
	&Or & 2 & 0 & 0 & 0 & 0 & 0 & 0 & 0 & 0 & \cellcolor{g} 0 & 2 \\ \hline
	&Sum& 418 & 295 & 12 & 126 & 104 & 78 & 18 & 2 & 0 & 0 & \\ \hline
\end{tabular}
\end{center}
\caption{Confusion matrix for trigger categories, top-down system.}
\label{pms_trigger_conf}
\end{sidewaystable}

\begin{table}
\begin{center}
\begin{tabular}{ | l | l | r | r | r | r | r | r | }
	\hline
	\multirow{8}{*}{True} & \multicolumn{7}{c|}{Predicted} \\ \cline{1-8}
	&\cellcolor{gray} & None & Agent & Theme & Co-theme & Part & Sum \\ \cline{2-8}
	&None & \cellcolor{g} 0 & 15 & 393 & 14 & 8 & 430 \\
	&Agent & 293 & \cellcolor{g} 6 & 5 & 0 & 0 & 304 \\
	&Theme & 1140 & 1 & \cellcolor{g} 260 & 3 & 0 & 1404 \\
	&Co-theme & 128 & 1 & 6 & \cellcolor{g} 8 & 0 & 143 \\
	&Part & 346 & 0 & 0 & 0 & \cellcolor{g} 8 & 354 \\ \hline	
	& Sum & 1907 & 23 & 664 & 25 & 16 & \\ \hline
\end{tabular}
\end{center}
\caption{Confusion matrix for argument categories, bottom-up system.}
\label{tees_argument_conf}
\end{table}

\begin{table}
\begin{center}
\begin{tabular}{ | l | l | r | r | r | r | r | r | }
	\hline
	\multirow{8}{*}{True} & \multicolumn{7}{c|}{Predicted} \\ \cline{1-8}
	&\cellcolor{gray} & None & Agent & Theme & Co-theme & Part & Sum \\ \cline{2-8}
	&None & \cellcolor{g} 0 & 14 & 185 & 2 & 0 &  301 \\
	&Agent & 57 & \cellcolor{g} 4 & 3 & 0 & 0 & 64 \\
	&Theme & 294 & 0 & \cellcolor{g} 140 & 0 & 0 & 434 \\
	&Co-theme & 55 & 0 & 0 & \cellcolor{g} 0 & 0 & 55 \\
	&Part & 94 & 0 & 0 & 0 & \cellcolor{g} 0 & 94 \\ \hline	
	& Sum & 500 & 18 & 328 & 2 & 0 & \\ \hline
\end{tabular}
\end{center}
\caption{Confusion matrix for argument categories, top-down system.}
\label{pms_argument_conf}
\end{table}
\fi

From these tables, precision, recall and f-score for each category for both systems can be calculated as presented in tables \ref{trigger_ev} and \ref{argument_ev}. 

\begin{table}
\begin{center}
\begin{tabular}{ | l | l | l | l | l | l | l | }
	\hline
	\cellcolor{gray} & \multicolumn{3}{c}{BOTTOM-UP} & \multicolumn{3}{c|}{TOP-DOWN} \\ \hline
	\cellcolor{gray} & Precision & Recall & F-score & Precision & Recall & F-Score \\ \hline
	Variable & 0.41 & 0.40 & 0.40 & 0.71 & 0.51 & 0.59 \\
	Thing & 0.33 & 0.14 & 0.20 & 0.5 & 0.22 & 0.31 \\
	Increase & 0.74 & 0.46 & 0.57 & 0.97 & 0.78 & 0.86 \\
	Decrease & 0.76 & 0.39 & 0.52 & 0.88 & 0.71 & 0.79 \\ 
	Change & 0.47 & 0.16 & 0.24 & 0.72 & 0.64 & 0.68 \\ 
	Cause & 0.11 & 0.01 & 0.02 & 0.72 & 0.33 & 0.45 \\ 
	Correlate & 0.68 & 0.13 & 0.22 & 1.00 & 0.01 & 0.02 \\ 
	And & 0.50 & 0.03 & 0.06 & 0.00 & 0.00 & 0.00 \\
	Or & 0.00 & 0.00 & 0.00 & 0.00 & 0.00 & 0.00 \\ \hline
	Macro & 0.44 & 0.25 & 0.32 & 0.61 & 0.35 & 0.44 \\
	Micro & 0.50 & 0.32 & 0.39 & 0.77 & 0.48 & 0.60 \\ \hline
\end{tabular}
\end{center}
\caption{Evaluation metrics, trigger categories.}
\label{trigger_ev}
\end{table}

\begin{table}
\begin{center}
\begin{tabular}{ | l | l | l | l | l | l | l | }	
	\hline
	\cellcolor{gray} & \multicolumn{3}{c}{BOTTOM-UP} & \multicolumn{3}{c|}{TOP-DOWN} \\ \hline
	\cellcolor{gray} & Precision & Recall & F-score & Precision & Recall & F-Score \\ \hline
	Agent & 0.26 & 0.02 & 0.04 & 0.22 & 0.06 & 0.09 \\
	Theme & 0.39 & 0.19 & 0.23 & 0.43 & 0.32 & 0.37 \\
	Co-theme & 0.32 & 0.06 & 0.10 & 0.00 & 0.00 & 0.00 \\
	Part & 0.80 & 0.02 & 0.04 & 0.00 & 0.00 & 0.00  \\ \hline
	Macro & 0.44 & 0.04 & 0.07 & 0.16 & 0.10 & 0.12 \\ 
	Micro & 0.43 & 0.13 & 0.20 & 0.31 & 0.22 & 0.26 \\ \hline
\end{tabular}
\end{center}
\caption{Evaluation metrics, argument categories.}
\label{argument_ev}
\end{table}
\section{Discussion}

It should be kept in mind that particularly the results produced by the top-down system provide a low estimate of the performance of a complete information extraction system on the task, as the pattern matching system used to provide the benchmark estimates is extremely primitive, and has a large potential for improvement. The bottom-up system used for the evaluation, TEES, has reached a mature step of development, and the results produced are therefore likely close to expected actual performance. However, the system has been developed for a different domain, and it is therefore likely that performance can be improved with a domain-tailored bottom-up system.

\todo[inline]{Maybe say something about data limitations and reliability of ML on the corpus?}

It can be seen from Table \ref{trigger_ev} that the top-down system outperforms bottom-up on the entity categories (Variable, Thing). As the bottom-up system starts with entity extraction, one would expect it to perform well on these categories, whereas in the top-down system, which treats entity extraction as a downstream task, one would expect errors from the event extraction step propagate to create problems for entity extraction. The results for these categories are therefore different for what would be expected, giving evidence that the extraction task at hand does not lend itself easily to the bottom-up approach. 

The top-down system outperformed the bottom-up system on all change event categories (Increase, Decrease and Change). For both systems, performance was higher on change event categories than on entity categories, or in fact, on any other trigger category group. This indicates that the change event categories are easier to recognize than the other categories in the annotation scheme used here.

For the interaction events, the top-down system clearly the bottom-up system on the \emph{cause} category, but, surprisingly, the opposite holds true for the \emph{correlation} category. Given the differences in performance on the change event categories, one would expect the top-down system to yield the best performance in both categories, but the top-down system yielded an extremely low recall on the correlation category. This is because, at the current state of development, technical issues prevented the development of high-coverage patterns for the correlation category.

The grammatical categories (And, Or) were not handled by the top-down pattern matching system, so no interesting comparison could be made for these categories.

Analysis of performance on argument categories does not bring any new insights: Performance on the \emph{theme} category is higher for the top-down system, due to the increase number of correct change events detected. The top-down system scores 0.00 for the \emph{co-theme} category, due to fact that the system at the current stage of development is unable to distinguish themes and co-themes in correlations, and the \emph{part} category is not detected, as the grammatical categories are not handled by the pattern matching system. It is surprising that the performance for the \emph{agent} category is rather similar for the two systems, especially given that the top-down system by far outperforms the bottom-up system on the cause category. However, not much effort had been put into the correct selection of arguments for the interaction categories during development of the pattern matching system, so it is likely that performance on this category will improve significantly in a more mature system.

Overall, it can be observed that the top-down system generally outperforms the bottom-up system, with some exceptions where the primitiveness of the prototype system drastically degrades performance. It is likely that much of the success of the top-down system is due to the fact that it started extraction from change events, which were shown to be the easiest categories, and used evidence found during this step to facilitate extraction of the more difficult categories. This can be contrasted with the bottom-up system, which started from the quite difficult entity categories, and the rather unreliable conclusions made during that step could not improve performance on the already easy change events categories.
\section{Discussion and conclusion}

It is likely that much of the success of the alternative pipeline is due to the fact that it started by extracting from change events, which appear to be the easiest categories. Evidence found during this step facilitates extraction of the more difficult categories. This can be contrasted with the traditional pipeline system, which started from the quite difficult entity categories, and the error-prone evidence found the that step could not improve performance on the already quite easy events categories. This result may possibly be generalized to the hypothesis that a pipeline information extraction system will attain the best results by starting extraction with the easiest category group.

The hypothesis is supported by the following probability theoretic argument: Information extraction can be formulated finding $argmax_{\boldsymbol{X}}\ P(X_0, X_1, ..., X_n | Y)$, where $X_0 ... X_n$ are the extraction item classes (entities, events and relations). Because it is ofen infeasable to finding argmax over the joint probablity distibution, the pipeline approach approximates the global optimum by decomposing the joint distribution using the chain rule of probability $P(X_0, X_1 ... X_n | Y) = P(X_0 | Y)P(X_1 | X_0, Y)P(X_2 | X_1, X_0, Y) \dots P(X_n | X_{n-1}, ..., X_0, Y)$ and greedily optimizing each step in the chain, thus computing $\hat{X_0} = argmax_{X_0}\ P(X_0 | Y)$, $\hat{X_1} = argmax_{X_1}\ P(X_1 | \hat{X_0}, Y)$ ... . In this setup, any error in $X_0$ will aversely affect all subsequent steps, as they then will try to optimize based on a suboptimal variable assignment. On the other hand, an error in $X_n$ will only affect the step itself. To minimize the approximation error, one should put the item classes that are least error-prone and least dependent on information from the other item classes first.

Verifying the hypothesis empirically, and exploring other heuristics for determining the optimal pipeline architecture for an extraction task remain tasks for future research. The development of an information extraction system that can lets the user make arbitrary adjustments to the pipeline would make this line of research immediately applicable. A longer term goal is the development of a system that is able to automatically select the optimal pipeline based directly on the data.

Running the extaction experiments has also given insight into the strengths and weaknesses of our annotation scheme, and refining the annotation scheme to sort out unnecessary complexities also remains an important research task. For instance, experiments in reasoning have uncovered that it does not make sense to make a binary distinction between entities that require no semantic interpretation (\emph{variables}) and entities that require full semantic interpretation (\emph{things}), because entities always require semantic interpretation to a certain degree. As an example, consider \emph{growth} in the example sentence "Increasing concentrations of CO2 cause  a strong decline in growth...". \emph{Growth} is arguably a quantitative variable, but for this to be useful during reasoning, a semantic interpretation component must disambiguate further what thing in the real world that the growth pertains to. In future versions of the annotation scheme, the category \emph{thing} will therefore be deprecated, and the category \emph{variable} used in all cases. All entities will then be passed to the semantic interpretation module.

THERE IS PLENTY OF FUTURE WORK THAT WE CAN MENTION HERE, AMONG OTHERS ONTOLOGIES, ADDING CO-REF TREATMENT, VARIABLE DISAMBIGUATION, REASONING, LDB ETC. IS IT WORTH MENTIONING, OR JUST NOISE FOR THE READER? 




%%%%%%%%%%%%%%%%%%%%%%%%%%%%%%%%%%%%%%%%%%%%%%
%%                                          %%
%% Backmatter begins here                   %%
%%                                          %%
%%%%%%%%%%%%%%%%%%%%%%%%%%%%%%%%%%%%%%%%%%%%%%

\begin{backmatter}

\section*{Competing interests}
  The authors declare that they have no competing interests.

\section*{Author's contributions}
    Text for this section \ldots

\section*{Acknowledgements}
  Much of the work presented in this article was conducted as a part of the Master's project of author Elias Aamot.
  
%%%%%%%%%%%%%%%%%%%%%%%%%%%%%%%%%%%%%%%%%%%%%%%%%%%%%%%%%%%%%
%%                  The Bibliography                       %%
%%                                                         %%
%%  Bmc_mathpys.bst  will be used to                       %%
%%  create a .BBL file for submission.                     %%
%%  After submission of the .TEX file,                     %%
%%  you will be prompted to submit your .BBL file.         %%
%%                                                         %%
%%                                                         %%
%%  Note that the displayed Bibliography will not          %%
%%  necessarily be rendered by Latex exactly as specified  %%
%%  in the online Instructions for Authors.                %%
%%                                                         %%
%%%%%%%%%%%%%%%%%%%%%%%%%%%%%%%%%%%%%%%%%%%%%%%%%%%%%%%%%%%%%

% if your bibliography is in bibtex format, use those commands:
\bibliographystyle{bmc-mathphys} % Style BST file
\bibliography{bib.bib}      % Bibliography file (usually '*.bib' )

\end{backmatter}
\end{document}
