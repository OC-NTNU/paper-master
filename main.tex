%% BioMed_Central_Tex_Template_v1.06
%%                                      %
%  bmc_article.tex            ver: 1.06 %
%                                       %

%%IMPORTANT: do not delete the first line of this template
%%It must be present to enable the BMC Submission system to
%%recognise this template!!

%%%%%%%%%%%%%%%%%%%%%%%%%%%%%%%%%%%%%%%%%
%%                                     %%
%%  LaTeX template for BioMed Central  %%
%%     journal article submissions     %%
%%                                     %%
%%          <8 June 2012>              %%
%%                                     %%
%%                                     %%
%%%%%%%%%%%%%%%%%%%%%%%%%%%%%%%%%%%%%%%%%


%%%%%%%%%%%%%%%%%%%%%%%%%%%%%%%%%%%%%%%%%%%%%%%%%%%%%%%%%%%%%%%%%%%%%
%%                                                                 %%
%% For instructions on how to fill out this Tex template           %%
%% document please refer to Readme.html and the instructions for   %%
%% authors page on the biomed central website                      %%
%% http://www.biomedcentral.com/info/authors/                      %%
%%                                                                 %%
%% Please do not use \input{...} to include other tex files.       %%
%% Submit your LaTeX manuscript as one .tex document.              %%
%%                                                                 %%
%% All additional figures and files should be attached             %%
%% separately and not embedded in the \TeX\ document itself.       %%
%%                                                                 %%
%% BioMed Central currently use the MikTex distribution of         %%
%% TeX for Windows) of TeX and LaTeX.  This is available from      %%
%% http://www.miktex.org                                           %%
%%                                                                 %%
%%%%%%%%%%%%%%%%%%%%%%%%%%%%%%%%%%%%%%%%%%%%%%%%%%%%%%%%%%%%%%%%%%%%%

%%% additional documentclass options:
%  [doublespacing]
%  [linenumbers]   - put the line numbers on margins

%%% loading packages, author definitions

%\documentclass[twocolumn]{bmcart}% uncomment this for twocolumn layout and comment line below
\documentclass{bmcart}

%%% Load packages
%\usepackage{amsthm,amsmath}
%\RequirePackage{natbib}
%\RequirePackage{hyperref}
\usepackage[utf8]{inputenc} %unicode support
%\usepackage[applemac]{inputenc} %applemac support if unicode package fails
%\usepackage[latin1]{inputenc} %UNIX support if unicode package fails

\usepackage{natbib}
\usepackage{hyperref}
\usepackage{graphicx}
\usepackage{multirow}
\usepackage{rotating}
\usepackage{xcolor,colortbl}
\usepackage[color=red]{todonotes}
\usepackage{qtree}
\usepackage{tikz-dependency}


%%%%%%%%%%%%%%%%%%%%%%%%%%%%%%%%%%%%%%%%%%%%%%%%%
%%                                             %%
%%  If you wish to display your graphics for   %%
%%  your own use using includegraphic or       %%
%%  includegraphics, then comment out the      %%
%%  following two lines of code.               %%
%%  NB: These line *must* be included when     %%
%%  submitting to BMC.                         %%
%%  All figure files must be submitted as      %%
%%  separate graphics through the BMC          %%
%%  submission process, not included in the    %%
%%  submitted article.                         %%
%%                                             %%
%%%%%%%%%%%%%%%%%%%%%%%%%%%%%%%%%%%%%%%%%%%%%%%%%


\def\includegraphic{}
\def\includegraphics{}



%%% Put your definitions there:
\startlocaldefs
\endlocaldefs


%%% Begin ...
\begin{document}

%%% Start of article front matter
\begin{frontmatter}

\begin{fmbox}
\dochead{Research}

%%%%%%%%%%%%%%%%%%%%%%%%%%%%%%%%%%%%%%%%%%%%%%
%%                                          %%
%% Enter the title of your article here     %%
%%                                          %%
%%%%%%%%%%%%%%%%%%%%%%%%%%%%%%%%%%%%%%%%%%%%%%

\title{Paper on Information Extraction and such}

%%%%%%%%%%%%%%%%%%%%%%%%%%%%%%%%%%%%%%%%%%%%%%
%%                                          %%
%% Enter the authors here                   %%
%%                                          %%
%% Specify information, if available,       %%
%% in the form:                             %%
%%   <key>={<id1>,<id2>}                    %%
%%   <key>=                                 %%
%% Comment or delete the keys which are     %%
%% not used. Repeat \author command as much %%
%% as required.                             %%
%%                                          %%
%%%%%%%%%%%%%%%%%%%%%%%%%%%%%%%%%%%%%%%%%%%%%%

\author[
   addressref={aff1},                   % id's of addresses, e.g. {aff1,aff2}
   corref={aff1},                       % id of corresponding address, if any
   noteref={n1},                        % id's of article notes, if any
   email={elias.aamot@gmail.com}   % email address
]{\inits{EAA}\fnm{Elias} \snm{Aamot}}
\author[
   addressref={aff1,aff2},
   email={john.RS.Smith@cambridge.co.uk}
]{\inits{XY}\fnm{X} \snm{Y}}

%%%%%%%%%%%%%%%%%%%%%%%%%%%%%%%%%%%%%%%%%%%%%%
%%                                          %%
%% Enter the authors' addresses here        %%
%%                                          %%
%% Repeat \address commands as much as      %%
%% required.                                %%
%%                                          %%
%%%%%%%%%%%%%%%%%%%%%%%%%%%%%%%%%%%%%%%%%%%%%%

\address[id=aff1]{%                           % unique id
  \orgname{Department of Zoology, Cambridge}, % university, etc
  \street{Waterloo Road},                     %
  %\postcode{}                                % post or zip code
  \city{London},                              % city
  \cny{UK}                                    % country
}
\address[id=aff2]{%
  \orgname{Marine Ecology Department, Institute of Marine Sciences Kiel},
  \street{D\"{u}sternbrooker Weg 20},
  \postcode{24105}
  \city{Kiel},
  \cny{Germany}
}

%%%%%%%%%%%%%%%%%%%%%%%%%%%%%%%%%%%%%%%%%%%%%%
%%                                          %%
%% Enter short notes here                   %%
%%                                          %%
%% Short notes will be after addresses      %%
%% on first page.                           %%
%%                                          %%
%%%%%%%%%%%%%%%%%%%%%%%%%%%%%%%%%%%%%%%%%%%%%%

\begin{artnotes}
%\note{Sample of title note}     % note to the article
\note[id=n1]{Equal contributor} % note, connected to author
\end{artnotes}

\end{fmbox}% comment this for two column layout

%%%%%%%%%%%%%%%%%%%%%%%%%%%%%%%%%%%%%%%%%%%%%%
%%                                          %%
%% The Abstract begins here                 %%
%%                                          %%
%% Please refer to the Instructions for     %%
%% authors on http://www.biomedcentral.com  %%
%% and include the section headings         %%
%% accordingly for your article type.       %%
%%                                          %%
%%%%%%%%%%%%%%%%%%%%%%%%%%%%%%%%%%%%%%%%%%%%%%

\begin{abstractbox}

\begin{abstract} % abstract
\parttitle{First part title} %if any
Text for this section.

\parttitle{Second part title} %if any
Text for this section.
\end{abstract}

%%%%%%%%%%%%%%%%%%%%%%%%%%%%%%%%%%%%%%%%%%%%%%
%%                                          %%
%% The keywords begin here                  %%
%%                                          %%
%% Put each keyword in separate \kwd{}.     %%
%%                                          %%
%%%%%%%%%%%%%%%%%%%%%%%%%%%%%%%%%%%%%%%%%%%%%%

\begin{keyword}
\kwd{sample}
\kwd{article}
\kwd{author}
\end{keyword}

% MSC classifications codes, if any
%\begin{keyword}[class=AMS]
%\kwd[Primary ]{}
%\kwd{}
%\kwd[; secondary ]{}
%\end{keyword}

\end{abstractbox}
%
%\end{fmbox}% uncomment this for twcolumn layout

\end{frontmatter}


%%%%%%%%%%%%%%%%%%%%%%%%%%%%%%%%%%%%%%%%%%%%%%
%% Main stuff
%%%%%%%%%%%%%%%%%%%%%%%%%%%%%%%%%%%%%%%%%%%%%%
\section{Introduction}

Information extraction is the task of extracting structured information from natural language text. Historically, information extraction research focused on named entity recognition (NER), with research effort slowly turning to more complex tasks such as relation or event extraction when state-of-the-art on NER had reached acceptable levels. 

When developing information extraction systems for a new domain, the development effort normally adheres to an incremental approach similar to historical development of the information extraction field, starting with named entity recognition, and subsequently working on relation or event extraction. The architecture of most state-of-the-art information extraction systems reflect this development pattern, as they are pipeline systems with first a NER component, followed by a relation/event extraction component, where the output of the first component is provided as input to the second component.

The drawback of a pipeline architecture is that each component must make a hard choice that cannot be undone by downstream components, even though new information may surface later in the extraction process. Some recent research effort has therefore focused on joint extraction architectures, where are steps are conducted in parallel.\todo{citations?} However, joint extraction architectures have not reached the level of maturity required for state-of-the-art performance, and have other drawbacks, such as the increased complexity of learning a joint probability distribution, that make them infeasible for certain circumstances\todo{Am I just making this stuff up, or is it true?}, such as when training data is scarce.

The traditional pipeline architecture for information extraction appears to have arisen not as a principled design decision, but rather accidentally by following the research effort. This paper therefore investigates another approach to the pipeline architecture, where events detection precedes entity detection, and shows that for certain extraction tasks, reversing the order of the pipeline can prove beneficial.  

\subsection{Extraction task}

Progress in research towards understanding the full range of causes and consequences of global warming is hampered by the wide range of relevant disciplines, which include, among others, climate science, earth science, oceanography, ecology and biochemistry. Text mining can be used to provide an overview of the research in the relevant disciplines, and help the researchers draw parallels or uncover hidden connections. As a first step towards a discovery support system in climate change science, an information extraction system is being developed.

To extract relations that are general enough to be useful across disciplines, but still specific enough to be useful for a researcher, the information extraction system targets the extraction of events where quantiative variables undergo a directional change, such as \emph{increase in atmospheric CO$_2$}, and interactions between such change events, which are restricted to causal and correlative relations. A pilot corpus consisting of 10 scientific journal articles has been annotated, as described in \citet{mar14}. The remainder of this section provides a sufficient introduction to the annotation scheme as to understand the specifics of the experiment.

The trigger categories that were used to annotate text spans of interest in the corpus are presented in Table \ref{ann_scheme}.

\begin{figure}
\Tree[.CATEGORY [.ENTITY \textit{Variable} \textit{Thing} ]
          [.EVENT [.\textit{Change} \textit{Increase} \textit{Decrease} ] 
         		  [.INTERACTION \textit{Cause} \textit{Correlate} ] 
          ] 
     ]
\caption{Type Hierarchy for the Annotation Scheme}
\label{ann_scheme}
\end{figure}

\emph{Variable} in our annotation scheme is defined as a \emph{quantitative variable}, meaning an entity than can be measured and assigned some value along some ordered axis, either as a numerical value or a value from a totally ordered set of discrete states. This includes among other counts, frequencies and ratios. Text spans that can be interpreted as quantitative variables out of context are abundant in the scientific literature, but only a small subset of these are of interest. Only variables that occur in an event are therefore annotated in the corpus. 

\emph{Thing} is used to annotate any span of text that functions as the argument of a change event, while not fulfilling the requirements to be annotated as a variable. 

A change event describes a directional, quantitative change in the value of a quantitative variable. \emph{Increase} is used to annotate a change in the positive direction, \emph{decrease} annotates changes in the negative direction and \emph{change} is used when the direction is underspecified in the text. All the change events take an entity as the \emph{theme} argument, signalling the variable that undergoes the change.

Interaction categories mark text spans that explicitly signal interactions between two change events. The types of interactions that are used in the corpus are \emph{cause} and \emph{correlate}. Cause events take two arguments: An \emph{agent}, which marks the change event that causes the other change event to come about. The \emph{theme} argument is used for the caused change event. Correlation events also take two arguments, the \emph{theme} and \emph{co-theme}. It was noticed during corpus annotation that there is a tendency in natural language to focus on one of the changes as initiating the other, giving such correlations a directional interpretation. For correlations with an directional interpretation, the initiating event is marked as the theme, and the other event is marked as the co-theme. If the correlation has no directional interpretation, the event that is the syntactic object of the correlation trigger is marked as co-theme, and the other event as theme by default.

Three additional categories are used in the corpus to handle some common language constructs, namely conjunction, disjunction and referring expressions. The categories \emph{and} and \emph{or} handle conjunctions and disjunctions, respectively, and each take two \emph{part} arguments. Referring expressions are handled by the \emph{RefExp} categories, which takes the antecedent as the \emph{coref} argument.

\subsection{Pipeline Architecture}

The canonical information extraction pipeline architecture uses a bottom-up approach, in which the lowest order categories (entities) are extracted first, and subsequently used as evidence during later steps when higher order categories (events) are extracted. However, it is also possible to conceive a top-down approach, where higher order categories are extracted first, and the lower level entities are extracted later. 

It was mentioned in the introduction that the bottom-up pipeline approach did not arise as a consequence of an informed design decision, but rather occurred accidentally due to the way research had been conducted. One hypothesis is that the bottom-up approach is successful for many extraction tasks because entity extraction is, in many tasks, a simpler sub-task than event extraction. Starting extraction from the simplest categories makes sense, as the performance on these categories is likely to be high, and the system therefore able to produce trustworthy information that can be used as supporting evidence during extraction of more difficult categories. A bolder hypothesis is that a pipeline approach will yield the best results if starting extraction from the simplest category group.

\section{Methods}

An evaluation was conducted to compare the performance of the two information extraction approaches on the specified task. Statistical information extraction systems traditionally follow the bottom-up approach, and an existing information exaction system was therefore selected to provide the benchmark for the bottom-up approach. On the other hand, the top-down approach is rarely taken in the literature. It was therefore necessary to develop a prototype top-down information extraction system for the sake of comparison.

To act as a benchmark bottom-up system, the Turku Event Extraction System (TEES) \citep{bjö11ddi} was selected. TEES was selected for the evaluation because it is has shown state-of-the-art performance over several years of development, can extract events with arbitrary structures, can be trained with custom data sets and the source code is publicly available\footnote{\url{https://github.com/jbjorne/TEES}}. TEES has been developed for BioNLP, and has attained high scores on a number of shared tasks, including BioNLP 2009 (1st place) \citep{bjö09}, BioNLP 2011 (1st place in 4/8 tasks) \citep{bjö11} and Drug-Drug Interactions 2011 Challenge (4th place) \citep{bjö11ddi}.

\todo[inline]{What should I write more on TEES?}

In order to benchmark the top-down approach, an information extraction system had to be developed that followed this approach. Inspection of the data revealed that trigger words for events were relatively unambiguous. It was therefore concluded that a deterministic pattern matching system would be sufficient to provide an estimate for the performance of the top-down approach on the task at hand.

In the system, a pattern is defined as a partial dependency graph that signals the presence of an event. During information extraction, the system parses the text with the Stanford parser\citep{kle03}, and an event is detected for every pattern that matches the dependency parse tree of the sentence. In case of a match, each pattern specifies which node(s) in the dependency tree should be labelled as the event trigger, and which parts of the dependency tree that should be labelled as the arguments. 

A set of patterns was manually developed from eight of the papers in the corpus, with the remaining two held out as test data. For every annotated event in the corpus, a pattern was written that would extract that event. To improve performance, each pattern was manually evaluated against the eight papers, and patterns that yielded a high ratio of false positives on the development data were excluded.

\todo[inline]{I am unsure how much detail is interesting about the PMS, and how to make it appear like we did not only do this because we were lazy.}

TEES was evaluated by five-fold cross-validation, each fold consisting of two papers. As eight papers had been used during development of the pattern matching system, the pattern matching system was evaluated once only, using the two remaining papers as test data.
\section{Results}

R
\section{Discussion}

It should be kept in mind that particularly the results produced by the top-down system provide a low estimate of the performance of a complete information extraction system on the task, as the pattern matching system used to provide the benchmark estimates is extremely primitive, and has a large potential for improvement. The bottom-up system used for the evaluation, TEES, has reached a mature step of development, and the results produced are therefore likely close to expected actual performance. However, the system has been developed for a different domain, and it is therefore likely that performance can be improved somewhat with a domain-tailored bottom-up system.

\todo[inline]{Maybe say something about data limitations and reliability of ML on the corpus?}

It can be seen from table \ref{trigger_ev} that the top-down system outperforms bottom-up on the entity categories (Variable, Thing). As the bottom-up system starts with entity extraction, one would expect it to perform well on these categories, whereas in the top-down system, which treats entity extraction as a secondary task, one would expect errors from the event extraction step propagate to create problems for entity extraction. The results for these categories are therefore different for what would be expected, giving evidence that the extraction task at hand does not lend itself easily to the bottom-up approach. 

The top-down system outperformed the bottom-up system on all change event categories (Increase, Decrease and Change). For both systems, performance was higher on change event categories than on entity categories, or in fact, on any other trigger category group. This indicates that the change event categories are easier to recognize than the other categories in the annotation scheme used here.

For the interaction events, the top-down system clearly the bottom-up system on the \emph{cause} category, but, surprisingly, the opposite holds true for the \emph{correlation} category. Given the differences in performance on the change event categories, one would expect the top-down system to yield the best performance in both categories, but the top-down system yielded an extremely low recall on the correlation category. This is because, at the current state of development, technical issues prevented the development of high-coverage patterns for the correlation category\todo{Do I have to go into detail?}. \todo{Do we want to analyse the performance of TEES on the categories further?}

The grammatical categories (And, Or) were not handled by the top-down pattern matching system, so no interesting comparison could be made for these categories.

Analysis of performance on argument categories does not bring any new insights: Performance on the \emph{theme} category is higher for the top-down system, due to the increase number of correct change events detected. The top-down system scores 0.00 for the \emph{co-theme} category, due to fact that the system at the current stage of development is unable to distinguish themes and co-themes in correlations, and the \emph{part} category is not detected, as the grammatical categories are not handled by the pattern matching system. It is surprising that the performance for the \emph{agent} category is rather similar for the two systems, especially given that the top-down system by far outperforms the bottom-up system on the cause category. However, not much effort had been put into the correct selection of arguments for the interaction categories during development of the pattern matching system, so it is likely that performance on this category will improve significantly in a more mature system.

Overall, it can be observed that the top-down system generally outperforms the bottom-up system, with some exceptions where the primitiveness of the prototype system drastically degrades performance. It is likely that much of the success of the top-down system is due to the fact that it started extraction from change events, which were shown to be the easiest categories, and used evidence found during this step to facilitate extraction of the more difficult categories. This can be contrasted with the bottom-up system, which started from the quite difficult entity categories, and the rather unreliable conclusions made during that step could not improve performance on the already easy change events categories. A general lesson that can be drawn from this, is to always start extraction from the easiest category types.
\section{Conclusion}

It appears that the canonical information extraction pipeline architecture has arisen by accident rather than as the result of a principled design decision, and that changing the order of pipeline components can yield significant improvements for certain domains without the additional complexity caused by joint extraction architectures.




%%%%%%%%%%%%%%%%%%%%%%%%%%%%%%%%%%%%%%%%%%%%%%
%%                                          %%
%% Backmatter begins here                   %%
%%                                          %%
%%%%%%%%%%%%%%%%%%%%%%%%%%%%%%%%%%%%%%%%%%%%%%

\begin{backmatter}

\section*{Competing interests}
  The authors declare that they have no competing interests.

\section*{Author's contributions}
    Text for this section \ldots

\section*{Acknowledgements}
  Much of the work presented in this article was conducted as a part of the Master's project of author Elias Aamot.
  
%%%%%%%%%%%%%%%%%%%%%%%%%%%%%%%%%%%%%%%%%%%%%%%%%%%%%%%%%%%%%
%%                  The Bibliography                       %%
%%                                                         %%
%%  Bmc_mathpys.bst  will be used to                       %%
%%  create a .BBL file for submission.                     %%
%%  After submission of the .TEX file,                     %%
%%  you will be prompted to submit your .BBL file.         %%
%%                                                         %%
%%                                                         %%
%%  Note that the displayed Bibliography will not          %%
%%  necessarily be rendered by Latex exactly as specified  %%
%%  in the online Instructions for Authors.                %%
%%                                                         %%
%%%%%%%%%%%%%%%%%%%%%%%%%%%%%%%%%%%%%%%%%%%%%%%%%%%%%%%%%%%%%

% if your bibliography is in bibtex format, use those commands:
\bibliographystyle{bmc-mathphys} % Style BST file
\bibliography{bib.bib}      % Bibliography file (usually '*.bib' )

\end{backmatter}
\end{document}
