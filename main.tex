%% BioMed_Central_Tex_Template_v1.06
%%                                      %
%  bmc_article.tex            ver: 1.06 %
%                                       %

%%IMPORTANT: do not delete the first line of this template
%%It must be present to enable the BMC Submission system to
%%recognise this template!!

%%%%%%%%%%%%%%%%%%%%%%%%%%%%%%%%%%%%%%%%%
%%                                     %%
%%  LaTeX template for BioMed Central  %%
%%     journal article submissions     %%
%%                                     %%
%%          <8 June 2012>              %%
%%                                     %%
%%                                     %%
%%%%%%%%%%%%%%%%%%%%%%%%%%%%%%%%%%%%%%%%%


%%%%%%%%%%%%%%%%%%%%%%%%%%%%%%%%%%%%%%%%%%%%%%%%%%%%%%%%%%%%%%%%%%%%%
%%                                                                 %%
%% For instructions on how to fill out this Tex template           %%
%% document please refer to Readme.html and the instructions for   %%
%% authors page on the biomed central website                      %%
%% http://www.biomedcentral.com/info/authors/                      %%
%%                                                                 %%
%% Please do not use \input{...} to include other tex files.       %%
%% Submit your LaTeX manuscript as one .tex document.              %%
%%                                                                 %%
%% All additional figures and files should be attached             %%
%% separately and not embedded in the \TeX\ document itself.       %%
%%                                                                 %%
%% BioMed Central currently use the MikTex distribution of         %%
%% TeX for Windows) of TeX and LaTeX.  This is available from      %%
%% http://www.miktex.org                                           %%
%%                                                                 %%
%%%%%%%%%%%%%%%%%%%%%%%%%%%%%%%%%%%%%%%%%%%%%%%%%%%%%%%%%%%%%%%%%%%%%

%%% additional documentclass options:
%  [doublespacing]
%  [linenumbers]   - put the line numbers on margins

%%% loading packages, author definitions

%\documentclass[twocolumn]{bmcart}% uncomment this for twocolumn layout and comment line below
\documentclass{bmcart}

%%% Load packages
\usepackage{amsthm,amsmath}
%\RequirePackage{natbib}
%\RequirePackage{hyperref}
\usepackage[utf8]{inputenc} %unicode support
%\usepackage[applemac]{inputenc} %applemac support if unicode package fails
%\usepackage[latin1]{inputenc} %UNIX support if unicode package fails

\usepackage{natbib}
\usepackage{hyperref}
\usepackage{graphicx}
\usepackage{multirow}
\usepackage{rotating}
%\usepackage{xcolor,colortbl}
%\usepackage[color=red]{todonotes}
%\usepackage{qtree}
%\usepackage{tikz-dependency}

\DeclareMathOperator*{\argmax}{\arg\!\max}


%%%%%%%%%%%%%%%%%%%%%%%%%%%%%%%%%%%%%%%%%%%%%%%%%
%%                                             %%
%%  If you wish to display your graphics for   %%
%%  your own use using includegraphic or       %%
%%  includegraphics, then comment out the      %%
%%  following two lines of code.               %%
%%  NB: These line *must* be included when     %%
%%  submitting to BMC.                         %%
%%  All figure files must be submitted as      %%
%%  separate graphics through the BMC          %%
%%  submission process, not included in the    %%
%%  submitted article.                         %%
%%                                             %%
%%%%%%%%%%%%%%%%%%%%%%%%%%%%%%%%%%%%%%%%%%%%%%%%%


\def\includegraphic{}
\def\includegraphics{}



%%% Put your definitions there:
\startlocaldefs
\endlocaldefs


%%% Begin ...
\begin{document}

%%% Start of article front matter
\begin{frontmatter}

\begin{fmbox}
\dochead{Research}

%%%%%%%%%%%%%%%%%%%%%%%%%%%%%%%%%%%%%%%%%%%%%%
%%                                          %%
%% Enter the title of your article here     %%
%%                                          %%
%%%%%%%%%%%%%%%%%%%%%%%%%%%%%%%%%%%%%%%%%%%%%%

\title{Customizing the information extraction pipeline: An experiment in text mining for climate science}

%%%%%%%%%%%%%%%%%%%%%%%%%%%%%%%%%%%%%%%%%%%%%%
%%                                          %%
%% Enter the authors here                   %%
%%                                          %%
%% Specify information, if available,       %%
%% in the form:                             %%
%%   <key>={<id1>,<id2>}                    %%
%%   <key>=                                 %%
%% Comment or delete the keys which are     %%
%% not used. Repeat \author command as much %%
%% as required.                             %%
%%                                          %%
%%%%%%%%%%%%%%%%%%%%%%%%%%%%%%%%%%%%%%%%%%%%%%

\author[
   addressref={aff1, aff2},                   % id's of addresses, e.g. {aff1,aff2}
   corref={aff1},                       % id of corresponding address, if any
   noteref={n1},                        % id's of article notes, if any
   email={eliasaa@google.com}   % email address
]{\inits{EAA}\fnm{Elias} \snm{Aamot}}
\author[
   addressref={aff1},
   email={john.RS.Smith@cambridge.co.uk}
]{\inits{XY}\fnm{X} \snm{Y}}

%%%%%%%%%%%%%%%%%%%%%%%%%%%%%%%%%%%%%%%%%%%%%%
%%                                          %%
%% Enter the authors' addresses here        %%
%%                                          %%
%% Repeat \address commands as much as      %%
%% required.                                %%
%%                                          %%
%%%%%%%%%%%%%%%%%%%%%%%%%%%%%%%%%%%%%%%%%%%%%%

\address[id=aff1]{%                           % unique id
  \orgname{Norwegian University of Science and Technology}, % university, etc
  %\street{Waterloo Road},                     %
  %\postcode{}                                % post or zip code
  \city{Trondheim},                              % city
  \cny{Norway}                                    % country
}
}

%%%%%%%%%%%%%%%%%%%%%%%%%%%%%%%%%%%%%%%%%%%%%%
%%                                          %%
%% Enter short notes here                   %%
%%                                          %%
%% Short notes will be after addresses      %%
%% on first page.                           %%
%%                                          %%
%%%%%%%%%%%%%%%%%%%%%%%%%%%%%%%%%%%%%%%%%%%%%%

\begin{artnotes}
%\note{Sample of title note}     % note to the article
\note[id=n1]{Equal contributor} % note, connected to author
\end{artnotes}

\end{fmbox}% comment this for two column layout

%%%%%%%%%%%%%%%%%%%%%%%%%%%%%%%%%%%%%%%%%%%%%%
%%                                          %%
%% The Abstract begins here                 %%
%%                                          %%
%% Please refer to the Instructions for     %%
%% authors on http://www.biomedcentral.com  %%
%% and include the section headings         %%
%% accordingly for your article type.       %%
%%                                          %%
%%%%%%%%%%%%%%%%%%%%%%%%%%%%%%%%%%%%%%%%%%%%%%

\begin{abstractbox}

\begin{abstract}

Information extraction can assist researchers by providing structured access to key findings. In traditional information extraction systems, entities are extracted in a first step, and events in a subsequent step. In this paper, we explore information extraction to support research in climate change science. The structure of the information extraction task poses challenges to the canonical pipeline. We therefore explore adapting the extraction pipeline to the task at hand, and show that such customizations can yield improvements for certain tasks.

\end{abstract}

%%%%%%%%%%%%%%%%%%%%%%%%%%%%%%%%%%%%%%%%%%%%%%
%%                                          %%
%% The keywords begin here                  %%
%%                                          %%
%% Put each keyword in separate \kwd{}.     %%
%%                                          %%
%%%%%%%%%%%%%%%%%%%%%%%%%%%%%%%%%%%%%%%%%%%%%%

\begin{keyword}
\kwd{information extraction}
\kwd{text mining}
\kwd{natural language processing}
\kwd{annotation}
\end{keyword}

% MSC classifications codes, if any
%\begin{keyword}[class=AMS]
%\kwd[Primary ]{}
%\kwd{}
%\kwd[; secondary ]{}
%\end{keyword}

\end{abstractbox}
%
%\end{fmbox}% uncomment this for twcolumn layout

\end{frontmatter}


%%%%%%%%%%%%%%%%%%%%%%%%%%%%%%%%%%%%%%%%%%%%%%
%% Main stuff
%%%%%%%%%%%%%%%%%%%%%%%%%%%%%%%%%%%%%%%%%%%%%%
\section{Introduction}

Information extraction is the task of extracting structured information from natural language text. Historically, information extraction research focused on named entity recognition (NER), with research effort slowly turning to more complex tasks such as relation or event extraction when state-of-the-art on NER had reached acceptable levels. 

When developing information extraction systems for a new domain, the development effort normally adheres to an incremental approach similar to historical development of the information extraction field, starting with named entity recognition, and subsequently working on relation or event extraction. The architecture of most state-of-the-art information extraction systems reflect this development pattern, as they are pipeline systems with first a NER component, followed by a relation/event extraction component, where the output of the first component is provided as input to the second component.

The drawback of a pipeline architecture is that each component must make a hard choice that cannot be undone by downstream components, even though new information may surface later in the extraction process. Some recent research effort has therefore focused on joint extraction architectures, where are steps are conducted in parallel.\todo{citations?} However, joint extraction architectures have not reached the level of maturity required for state-of-the-art performance, and have other drawbacks, such as the increased complexity of learning a joint probability distribution, that make them infeasible for certain circumstances\todo{Am I just making this stuff up, or is it true?}, such as when training data is scarce.

The traditional pipeline architecture for information extraction appears to have arisen not as a principled design decision, but rather accidentally by following the research effort. This paper therefore investigates another approach to the pipeline architecture, where events detection precedes entity detection, and shows that for certain extraction tasks, reversing the order of the pipeline can prove beneficial.  

\subsection{Extraction task}

Progress in research towards understanding the full range of causes and consequences of global warming is hampered by the wide range of relevant disciplines, which include, among others, climate science, earth science, oceanography, ecology and biochemistry. Text mining can be used to provide an overview of the research in the relevant disciplines, and help the researchers draw parallels or uncover hidden connections. As a first step towards a discovery support system in climate change science, an information extraction system is being developed.

To extract relations that are general enough to be useful across disciplines, but still specific enough to be useful for a researcher, the information extraction system targets the extraction of events where quantiative variables undergo a directional change, such as \emph{increase in atmospheric CO$_2$}, and interactions between such change events, which are restricted to causal and correlative relations. A pilot corpus consisting of 10 scientific journal articles has been annotated, as described in \citet{mar14}. The remainder of this section provides a sufficient introduction to the annotation scheme as to understand the specifics of the experiment.

The trigger categories that were used to annotate text spans of interest in the corpus are presented in Table \ref{ann_scheme}.

\begin{figure}
\Tree[.CATEGORY [.ENTITY \textit{Variable} \textit{Thing} ]
          [.EVENT [.\textit{Change} \textit{Increase} \textit{Decrease} ] 
         		  [.INTERACTION \textit{Cause} \textit{Correlate} ] 
          ] 
     ]
\caption{Type Hierarchy for the Annotation Scheme}
\label{ann_scheme}
\end{figure}

\emph{Variable} in our annotation scheme is defined as a \emph{quantitative variable}, meaning an entity than can be measured and assigned some value along some ordered axis, either as a numerical value or a value from a totally ordered set of discrete states. This includes among other counts, frequencies and ratios. Text spans that can be interpreted as quantitative variables out of context are abundant in the scientific literature, but only a small subset of these are of interest. Only variables that occur in an event are therefore annotated in the corpus. 

\emph{Thing} is used to annotate any span of text that functions as the argument of a change event, while not fulfilling the requirements to be annotated as a variable. 

A change event describes a directional, quantitative change in the value of a quantitative variable. \emph{Increase} is used to annotate a change in the positive direction, \emph{decrease} annotates changes in the negative direction and \emph{change} is used when the direction is underspecified in the text. All the change events take an entity as the \emph{theme} argument, signalling the variable that undergoes the change.

Interaction categories mark text spans that explicitly signal interactions between two change events. The types of interactions that are used in the corpus are \emph{cause} and \emph{correlate}. Cause events take two arguments: An \emph{agent}, which marks the change event that causes the other change event to come about. The \emph{theme} argument is used for the caused change event. Correlation events also take two arguments, the \emph{theme} and \emph{co-theme}. It was noticed during corpus annotation that there is a tendency in natural language to focus on one of the changes as initiating the other, giving such correlations a directional interpretation. For correlations with an directional interpretation, the initiating event is marked as the theme, and the other event is marked as the co-theme. If the correlation has no directional interpretation, the event that is the syntactic object of the correlation trigger is marked as co-theme, and the other event as theme by default.

Three additional categories are used in the corpus to handle some common language constructs, namely conjunction, disjunction and referring expressions. The categories \emph{and} and \emph{or} handle conjunctions and disjunctions, respectively, and each take two \emph{part} arguments. Referring expressions are handled by the \emph{RefExp} categories, which takes the antecedent as the \emph{coref} argument.

\subsection{Pipeline Architecture}

The canonical information extraction pipeline architecture uses a bottom-up approach, in which the lowest order categories (entities) are extracted first, and subsequently used as evidence during later steps when higher order categories (events) are extracted. However, it is also possible to conceive a top-down approach, where higher order categories are extracted first, and the lower level entities are extracted later. 

It was mentioned in the introduction that the bottom-up pipeline approach did not arise as a consequence of an informed design decision, but rather occurred accidentally due to the way research had been conducted. One hypothesis is that the bottom-up approach is successful for many extraction tasks because entity extraction is, in many tasks, a simpler sub-task than event extraction. Starting extraction from the simplest categories makes sense, as the performance on these categories is likely to be high, and the system therefore able to produce trustworthy information that can be used as supporting evidence during extraction of more difficult categories. A bolder hypothesis is that a pipeline approach will yield the best results if starting extraction from the simplest category group.

\section{Text mining in the climate change domain}

Research progress in the field of climate change is hampered by the wide range of relevant disciplines, which include, among others, climate science, earth science, oceanography, ecology and biochemistry. While a researcher may be able to keep up with the state-of-the-art within his/her field of specialization, it is impossible to keep track of all potentially useful findings in the related disciplines. Text mining can in this regards support the research effort by extracting structured and indexed representations of the findings in all relevant disciplines. This paper represents a line of work towards an information extraction systems to support scientific research in the domain of climate change science.

Climate change science aims to understand the causes and effects of climate change. The knowledge in the field can to a large extent be modelled as causal or correlative relations between variables, for instance \emph{increase in atmospheric CO$_2$ causes decrease in oceanic p.H.} or \emph{decrease in mineral nutrients correlates with decrease in phytoplankon abundance}. Our hypothesis is that this representional scheme is generic enough to represent fact accross all the relevant disciplines, yet specific enough to be useful to researchers in the domain. One observation that support this hypothesis is that this representational scheme is expressive enough to support inference of more complex structures, such as feedback loops \textemdash situations in which a change to a variable sets in motion a chain of events that, in turn, cause a new change to the original variable. Feedback loops are of particular interest to researchers in the field, due to the serious consequences of a potential feedback loop. 

This paper deals with automatic annotation rather than full information extraction. Annotation is the process of marking relevant text spans with category labels, and determining which relations hold between the marked text spans. For each task, the annotation guidelines specifies the set of category labels and relations which are used, and how they are used. The goal of automatic annotation is to automatically annotate unseen text with human-like annotation quality. To bridge the gap between automatic annotation and full information extraction, a post-processing procedure extracts the annotations and stores them as structured data. 

The annotation scheme used in the climate change science domain is described in detail in \cite{}. The remainder of this section will present the most important aspects of the annotation scheme. As an entry point, consider the following text fragment, taken from \cite{wal13} (some parentheses removed for clarity):

\begin{quote}

Increasing concentrations of CO2 cause a strong decline in growth, which decreases by up to 53\% over the investigated CO2 range.
Although the total carbon quota (TPC) is not affected by CO2, the organic carbon quota (POC) gradually increases while the inorganic carbon quota (PIC) shows a substantial decrease.

\end{quote}


The desired annotation for the sentences under the annotation scheme is presented in Figure [TODO: Erwin, I need your help for this]. The general annotation framework consists of two primitives: \emph{Triggers} and \emph{arguments}. Triggers are used to mark text spans that express interesting information of a specific type. Each trigger belongs to a category, taken from the categories presented in Table \ref{ann_scheme}. The usage of each category will be explained below. Arguments are directed links that mark relations between trigger spans. Arguments are typed, and certain types of trigger categories are required to take certain argument types, as explained below.

\begin{figure}
\Tree[.CATEGORY [.ENTITY \textit{Variable} \textit{Thing} ]
          [.EVENT [.\textit{Change} \textit{Increase} \textit{Decrease} ] 
         		  [.INTERACTION \textit{Cause} \textit{Correlate} ]
          ] 
     ]
\caption{Type Hierarchy of triggers in the Annotation Scheme}
\label{ann_scheme}
\end{figure}

\emph{Variable} is in our annotation scheme defined as a \emph{quantitative variable}, meaning an entity than can be measured and assigned some value along some ordered axis, either as a numerical value or a value from a totally ordered set of discrete states. This includes among other counts, frequencies and ratios. Examples of variables from the corpus include $CO_2$, \emph{organic carbon quota}, \emph{surface ocean pH} and \emph{mean light intensities}. Scientific articles normally contain a wide range of variables, but our annotation scheme limits itself to quantitative variables that occur in an event of interest, as only these can be used for making inferences by the downstream components. 

\emph{Thing} is used to annotate any span of text that functions as the argument of an event, but does not fulfil the requirements for being annotated as a variable, as described above. This occurs in phrases such as ''\emph{down-regulation of the gene}'', where \emph{down-regulation} clearly signals a quantitative change, and the only possible explicit argument of the change is \emph{the gene}. However, \emph{the gene} is not a variable. The variable that changes is the implicit \emph{activity level of the gene}, under the view that \emph{down-regulate} can be paraphrased as ''\emph{decrease the activity level of}''. The explicit argument is therefore annotated as a \emph{thing} rather than a \emph{variable}.

The underlying motivation for maintaining \emph{thing} and \emph{variable} as separate categories, is that the category \emph{thing} signals to the post-processing procedure that additional semantic interpretation should be factored in from the event trigger in order to specify the variable in question, as in the \emph{down-regulate} example. 

The change event categories are used to mark text spans that describe a directional, quantitative change in the value of a quantiative variable. \emph{Increase} is used to annotate a change in the positive direction, \emph{decrease} annotates changes in the negative direction and \emph{change} is used when the direction of the change is underspecified in the text. The change event takes the entity that undergoes the change as a \emph{theme} argument. A change event is not annotated unless an explicit entity that undergoes the signalled change can be inferred from the text.

Interaction categories mark text spans that explicitly signal interactions between two change events. The types of interactions that are used in the corpus are \emph{cause} and \emph{correlate}. Cause events take two arguments: An \emph{agent}, which marks the change event that causes the other change event to come about. The \emph{theme} argument is used for the caused change event. Correlation events also take two arguments, the \emph{theme} and \emph{co-theme}. It was noticed during corpus annotation that there is a tendency in natural language to focus on one of the changes as initiating the other, giving such correlations a directional interpretation. For correlations with an directional interpretation, the initiating event is marked as the theme, and the other event is marked as the co-theme. If the correlation has no directional interpretation, the event that is the syntactic object of the correlation trigger is marked as co-theme, and the other event as theme by default.

Sometimes an interaction holds between a change event and an entity or between two entities, rather than between two change events. An example of this is shown in the annotation of ''\emph{Although the total carbon quota (TPC) is not affected by CO2 \dots}''. Such entities are interpreted as the arguments of an implicit \emph{change}. 

It is not uncommon that the trigger word for a change event is used causatively, as in the sentence ''\emph{Heightened atmospheric CO$_2$ levels increase global temperatures}'', where \emph{increase} both signals an increase in the variable \emph{global temperatures} and signals a causal relationship between ''\emph{heightened atmospheric CO$_2$ levels}'' and ''\emph{increased global temperatures}''. In such cases, both change events are annotated normally, but in addition, the causative change event takes the other change event as an \emph{agent} argument, as if it were an interaction event.

One aspect in which our annotation scheme differs from the types of annotation schemes  traditionally used in biomedical corpora, such as GENIA\citep{kim08}, is that biomedical corpora normally annotate entities with types from a rich domain ontology, whereas our annotation scheme only subtypes entities into the broad categories \emph{variable} and \emph{thing}. The main reason for this is the lack of onotologies that cover all the disciplines related to climate change science.

\section{Methods}

An evaluation was conducted to compare the performance of the two information extraction approaches on the specified task. Statistical information extraction systems traditionally follow the bottom-up approach, and an existing information exaction system was therefore selected to provide the benchmark for the bottom-up approach. On the other hand, the top-down approach is rarely taken in the literature. It was therefore necessary to develop a prototype top-down information extraction system for the sake of comparison.

To act as a benchmark bottom-up system, the Turku Event Extraction System (TEES) \citep{bjö11ddi} was selected. TEES was selected for the evaluation because it is has shown state-of-the-art performance over several years of development, can extract events with arbitrary structures, can be trained with custom data sets and the source code is publicly available\footnote{\url{https://github.com/jbjorne/TEES}}. TEES has been developed for BioNLP, and has attained high scores on a number of shared tasks, including BioNLP 2009 (1st place) \citep{bjö09}, BioNLP 2011 (1st place in 4/8 tasks) \citep{bjö11} and Drug-Drug Interactions 2011 Challenge (4th place) \citep{bjö11ddi}.

\todo[inline]{What should I write more on TEES?}

In order to benchmark the top-down approach, an information extraction system had to be developed that followed this approach. Inspection of the data revealed that trigger words for events were relatively unambiguous. It was therefore concluded that a deterministic pattern matching system would be sufficient to provide an estimate for the performance of the top-down approach on the task at hand.

In the system, a pattern is defined as a partial dependency graph that signals the presence of an event. During information extraction, the system parses the text with the Stanford parser\citep{kle03}, and an event is detected for every pattern that matches the dependency parse tree of the sentence. In case of a match, each pattern specifies which node(s) in the dependency tree should be labelled as the event trigger, and which parts of the dependency tree that should be labelled as the arguments. 

A set of patterns was manually developed from eight of the papers in the corpus, with the remaining two held out as test data. For every annotated event in the corpus, a pattern was written that would extract that event. To improve performance, each pattern was manually evaluated against the eight papers, and patterns that yielded a high ratio of false positives on the development data were excluded.

\todo[inline]{I am unsure how much detail is interesting about the PMS, and how to make it appear like we did not only do this because we were lazy.}

TEES was evaluated by five-fold cross-validation, each fold consisting of two papers. As eight papers had been used during development of the pattern matching system, the pattern matching system was evaluated once only, using the two remaining papers as test data.
\section{Experiments}

This section presents the experiments. 

\subsection{Test data}

The annotated corpus consists of 10 full length scientific papers that were selected by a domain expert as represenative for the domain. The methods and materials section of the papers was not annotated, because this section consistently failed to yield any relevant information. All papers were taken from the open access journal PLOS ONE\footnote{\url{www.plosone.org}}, in order to avoid copyright issues when sharing the annotated corpus and any extracted material. 

The annotation itself was conducted using the brat rapid annotation tool\ref{ste12}. All 10 papers were annotated by author EAA, and subsequently reviewed by author EM. All disagreements were solved through discussion. Additional papers have been added to the corpus since the conclusion of the experiment presented here. The corpus is publicly available in its entirety through GitHub\footnote{\url{https://github.com/OC-NTNU/OCC}}.

\subsection{Evaluation methods}

TEES was evaluated by five-fold cross-validation, each fold consisting of two papers. As eight papers had been used during development of the pattern matching system, the pattern matching system was evaluated once only, using the two remaining papers as test data.

In the evaluation, an element is matched to an element in the gold standard if the text spans exhibit any degree of overlap, thus abstracting away from exact scoping of text spans. This was done to not penalize TEES for its biomedical-centric approach to extracting multiword expressions. A predicted trigger was counted as a true positive if it matched an element of the same annotation category in the gold standard, and a false positive if it didn't. A trigger in the gold standard was counted as a false negative if it could not be matched to a predicted trigger of the same category. 

\section{Results}

Precision, recall and f-score for each category for both systems were calculated and presented in Tables \ref{trigger_ev} and Table \ref{argument_ev}. 

\begin{table}
\begin{center}
\begin{tabular}{ | l | l | l | l | l | l | l | }
	\hline
	\cellcolor{gray} & \multicolumn{3}{c}{TEES} & \multicolumn{3}{c|}{PMS} \\ \hline
	\cellcolor{gray} & Precision & Recall & F-score & Precision & Recall & F-Score \\ \hline
	Variable & 0.41 & 0.40 & 0.40 & 0.71 & 0.51 & 0.59 \\
	Thing & 0.33 & 0.14 & 0.20 & 0.5 & 0.22 & 0.31 \\
	Increase & 0.74 & 0.46 & 0.57 & 0.97 & 0.78 & 0.86 \\
	Decrease & 0.76 & 0.39 & 0.52 & 0.88 & 0.71 & 0.79 \\ 
	Change & 0.47 & 0.16 & 0.24 & 0.72 & 0.64 & 0.68 \\ 
	Cause & 0.11 & 0.01 & 0.02 & 0.72 & 0.33 & 0.45 \\ 
	Correlate & 0.68 & 0.13 & 0.22 & 1.00 & 0.01 & 0.02 \\ \hline
	Macro & 0.50 & 0.24 & 0.31 & 0.79 & 0.46 & 0.53 \\
	Micro & 0.50 & 0.33 & 0.38 & 0.81 & 0.51 & 0.58 \\ \hline
\end{tabular}
\end{center}
\caption{Evaluation metrics, trigger categories.}
\label{trigger_ev}
\end{table}

\begin{table}
\begin{center}
\begin{tabular}{ | l | l | l | l | l | l | l | }	
	\hline
	\cellcolor{gray} & \multicolumn{3}{c}{TEES} & \multicolumn{3}{c|}{PMS} \\ \hline
	\cellcolor{gray} & Precision & Recall & F-score & Precision & Recall & F-Score \\ \hline
	Agent & 0.26 & 0.02 & 0.04 & 0.22 & 0.06 & 0.09 \\
	Theme & 0.39 & 0.19 & 0.23 & 0.43 & 0.32 & 0.37 \\
	Co-theme & 0.32 & 0.06 & 0.10 & 0.00 & 0.00 & 0.00 \\ \hline
	Macro & 0.32 & 0.09 & 0.12 & 0.22 & 0.13 & 0.15 \\ 
	Micro & 0.36 & 0.14 & 0.18 & 0.36 & 0.26 & 0.30 \\ \hline
\end{tabular}
\end{center}
\caption{Evaluation metrics, argument categories.}
\label{argument_ev}
\end{table}

It should be kept in mind that particularly the results produced provide a low estimate of the performance of a mature information extraction system on the task. The pattern matching system used to provide the benchmark estimates is extremely primitive, and has a large potential for improvement, whereas TEES, being a machine learning based system, likely suffers from lack of training data.

It can be seen from Table \ref{trigger_ev} that the PMS outperforms TEES on the entity categories (\emph{variable}, \emph{thing}). This seems to confirm that the additional information provided by having extracted change events yields benefits that far outweight the error propagted from the earlier step.

For both systems, performance was higher on the change event categories (\emph{increase}, \emph{decrease} and \emph{change}) than on the entity categories, or in fact, on any other trigger cateogry group. This indicates that the change event categories is the group of categories that is easiest to recognize correctly in this extraction task. The PMS outperformed TEES on all change event categories. It is likely tha the performance of TEES is affected negatively by errors in the earlier step, as the features used for event trigger detection include the number of entities in the sentence and entities within a linear window of the word in question.

For the interaction events, the PMS clearly outperforms the TEES system on the \emph{cause} category. However, the opposite holds true for the \emph{correlation} category. This is most likely due to the fact that the most productive patterns for correlations were removed to avoid false positives.

Analysis of performance on argument categories does not bring any new insights: Performance on the \emph{theme} category is higher for the PMS, due to the increased number of correct change events detected. The PMS scores 0.00 for the \emph{co-theme} category, due to fact that the system at the current stage of development is unable to properly distinguish themes and co-themes in correlations. It is surprising that the performance for the \emph{agent} category is rather similar for the two systems, especially given that the PMS by far outperforms TEES on the cause category. A significant reason for this is likely the fact that the PMS cannot handle causative change events at current stage of development.

Overall, the PMS system generally outperforms TTES system. The results are not perfect due to data limitation for TEES and system primitivity for the PMS, and due to the fact that the two systems take different approaches to information extraction, but the results still show that there could be some merit to altering the traditional information extraction pipeline.


\section{Conclusion}

It appears that the canonical information extraction pipeline architecture has arisen by accident rather than as the result of a principled design decision, and that changing the order of pipeline components can yield significant improvements for certain domains without the additional complexity caused by joint extraction architectures.




%%%%%%%%%%%%%%%%%%%%%%%%%%%%%%%%%%%%%%%%%%%%%%
%%                                          %%
%% Backmatter begins here                   %%
%%                                          %%
%%%%%%%%%%%%%%%%%%%%%%%%%%%%%%%%%%%%%%%%%%%%%%

\begin{backmatter}

\section*{Competing interests}
  The authors declare that they have no competing interests.

\section*{Author's contributions}
    	Text for this section \ldots

\section*{Acknowledgements}
  Much of the work presented in this article was conducted as a part of the Master's project of author Elias Aamot. Elias Aamot is currently affiliated with Google, Inc., but the work presented here was substantially conducted while affiliated with the Norwegian University of Science and Technology.
  
%%%%%%%%%%%%%%%%%%%%%%%%%%%%%%%%%%%%%%%%%%%%%%%%%%%%%%%%%%%%%
%%                  The Bibliography                       %%
%%                                                         %%
%%  Bmc_mathpys.bst  will be used to                       %%
%%  create a .BBL file for submission.                     %%
%%  After submission of the .TEX file,                     %%
%%  you will be prompted to submit your .BBL file.         %%
%%                                                         %%
%%                                                         %%
%%  Note that the displayed Bibliography will not          %%
%%  necessarily be rendered by Latex exactly as specified  %%
%%  in the online Instructions for Authors.                %%
%%                                                         %%
%%%%%%%%%%%%%%%%%%%%%%%%%%%%%%%%%%%%%%%%%%%%%%%%%%%%%%%%%%%%%

% if your bibliography is in bibtex format, use those commands:
\bibliographystyle{bmc-mathphys} % Style BST file
\bibliography{bib.bib}      % Bibliography file (usually '*.bib' )

\end{backmatter}
\end{document}
