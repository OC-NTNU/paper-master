%% BioMed_Central_Tex_Template_v1.06
%%                                      %
%  bmc_article.tex            ver: 1.06 %
%                                       %

%%IMPORTANT: do not delete the first line of this template
%%It must be present to enable the BMC Submission system to
%%recognise this template!!

%%%%%%%%%%%%%%%%%%%%%%%%%%%%%%%%%%%%%%%%%
%%                                     %%
%%  LaTeX template for BioMed Central  %%
%%     journal article submissions     %%
%%                                     %%
%%          <8 June 2012>              %%
%%                                     %%
%%                                     %%
%%%%%%%%%%%%%%%%%%%%%%%%%%%%%%%%%%%%%%%%%


%%%%%%%%%%%%%%%%%%%%%%%%%%%%%%%%%%%%%%%%%%%%%%%%%%%%%%%%%%%%%%%%%%%%%
%%                                                                 %%
%% For instructions on how to fill out this Tex template           %%
%% document please refer to Readme.html and the instructions for   %%
%% authors page on the biomed central website                      %%
%% http://www.biomedcentral.com/info/authors/                      %%
%%                                                                 %%
%% Please do not use \input{...} to include other tex files.       %%
%% Submit your LaTeX manuscript as one .tex document.              %%
%%                                                                 %%
%% All additional figures and files should be attached             %%
%% separately and not embedded in the \TeX\ document itself.       %%
%%                                                                 %%
%% BioMed Central currently use the MikTex distribution of         %%
%% TeX for Windows) of TeX and LaTeX.  This is available from      %%
%% http://www.miktex.org                                           %%
%%                                                                 %%
%%%%%%%%%%%%%%%%%%%%%%%%%%%%%%%%%%%%%%%%%%%%%%%%%%%%%%%%%%%%%%%%%%%%%

%%% additional documentclass options:
%  [doublespacing]
%  [linenumbers]   - put the line numbers on margins

%%% loading packages, author definitions

%\documentclass[twocolumn]{bmcart}% uncomment this for twocolumn layout and comment line below
\documentclass{bmcart}

%%% Load packages
\usepackage{amsthm,amsmath}
%\RequirePackage{natbib}
%\RequirePackage{hyperref}
\usepackage[utf8]{inputenc} %unicode support
%\usepackage[applemac]{inputenc} %applemac support if unicode package fails
%\usepackage[latin1]{inputenc} %UNIX support if unicode package fails

\usepackage{natbib}
\usepackage{hyperref}
\usepackage{graphicx}
\usepackage{multirow}
\usepackage{rotating}
%\usepackage{xcolor,colortbl}
%\usepackage[color=red]{todonotes}
%\usepackage{qtree}
%\usepackage{tikz-dependency}

\DeclareMathOperator*{\argmax}{\arg\!\max}


%%%%%%%%%%%%%%%%%%%%%%%%%%%%%%%%%%%%%%%%%%%%%%%%%
%%                                             %%
%%  If you wish to display your graphics for   %%
%%  your own use using includegraphic or       %%
%%  includegraphics, then comment out the      %%
%%  following two lines of code.               %%
%%  NB: These line *must* be included when     %%
%%  submitting to BMC.                         %%
%%  All figure files must be submitted as      %%
%%  separate graphics through the BMC          %%
%%  submission process, not included in the    %%
%%  submitted article.                         %%
%%                                             %%
%%%%%%%%%%%%%%%%%%%%%%%%%%%%%%%%%%%%%%%%%%%%%%%%%


\def\includegraphic{}
\def\includegraphics{}



%%% Put your definitions there:
\startlocaldefs
\endlocaldefs


%%% Begin ...
\begin{document}

%%% Start of article front matter
\begin{frontmatter}

\begin{fmbox}
\dochead{Research}

%%%%%%%%%%%%%%%%%%%%%%%%%%%%%%%%%%%%%%%%%%%%%%
%%                                          %%
%% Enter the title of your article here     %%
%%                                          %%
%%%%%%%%%%%%%%%%%%%%%%%%%%%%%%%%%%%%%%%%%%%%%%

\title{Customizing the information extraction pipeline: An experiment in text mining for climate science}

%%%%%%%%%%%%%%%%%%%%%%%%%%%%%%%%%%%%%%%%%%%%%%
%%                                          %%
%% Enter the authors here                   %%
%%                                          %%
%% Specify information, if available,       %%
%% in the form:                             %%
%%   <key>={<id1>,<id2>}                    %%
%%   <key>=                                 %%
%% Comment or delete the keys which are     %%
%% not used. Repeat \author command as much %%
%% as required.                             %%
%%                                          %%
%%%%%%%%%%%%%%%%%%%%%%%%%%%%%%%%%%%%%%%%%%%%%%

\author[
   addressref={aff1, aff2},                   % id's of addresses, e.g. {aff1,aff2}
   corref={aff1},                       % id of corresponding address, if any
   noteref={n1},                        % id's of article notes, if any
   email={eliasaa@google.com}   % email address
]{\inits{EAA}\fnm{Elias} \snm{Aamot}}
\author[
   addressref={aff1},
   email={john.RS.Smith@cambridge.co.uk}
]{\inits{XY}\fnm{X} \snm{Y}}

%%%%%%%%%%%%%%%%%%%%%%%%%%%%%%%%%%%%%%%%%%%%%%
%%                                          %%
%% Enter the authors' addresses here        %%
%%                                          %%
%% Repeat \address commands as much as      %%
%% required.                                %%
%%                                          %%
%%%%%%%%%%%%%%%%%%%%%%%%%%%%%%%%%%%%%%%%%%%%%%

\address[id=aff1]{%                           % unique id
  \orgname{Norwegian University of Science and Technology}, % university, etc
  %\street{Waterloo Road},                     %
  %\postcode{}                                % post or zip code
  \city{Trondheim},                              % city
  \cny{Norway}                                    % country
}
}

%%%%%%%%%%%%%%%%%%%%%%%%%%%%%%%%%%%%%%%%%%%%%%
%%                                          %%
%% Enter short notes here                   %%
%%                                          %%
%% Short notes will be after addresses      %%
%% on first page.                           %%
%%                                          %%
%%%%%%%%%%%%%%%%%%%%%%%%%%%%%%%%%%%%%%%%%%%%%%

\begin{artnotes}
%\note{Sample of title note}     % note to the article
\note[id=n1]{Equal contributor} % note, connected to author
\end{artnotes}

\end{fmbox}% comment this for two column layout

%%%%%%%%%%%%%%%%%%%%%%%%%%%%%%%%%%%%%%%%%%%%%%
%%                                          %%
%% The Abstract begins here                 %%
%%                                          %%
%% Please refer to the Instructions for     %%
%% authors on http://www.biomedcentral.com  %%
%% and include the section headings         %%
%% accordingly for your article type.       %%
%%                                          %%
%%%%%%%%%%%%%%%%%%%%%%%%%%%%%%%%%%%%%%%%%%%%%%

\begin{abstractbox}

\begin{abstract}

Information extraction can assist researchers by providing structured access to key findings. In traditional information extraction systems, entities are extracted in a first step, and events in a subsequent step. In this paper, we explore information extraction to support research in climate change science. The structure of the information extraction task poses challenges to the canonical pipeline. We therefore explore adapting the extraction pipeline to the task at hand, and show that such customizations can yield improvements for certain tasks.

\end{abstract}

%%%%%%%%%%%%%%%%%%%%%%%%%%%%%%%%%%%%%%%%%%%%%%
%%                                          %%
%% The keywords begin here                  %%
%%                                          %%
%% Put each keyword in separate \kwd{}.     %%
%%                                          %%
%%%%%%%%%%%%%%%%%%%%%%%%%%%%%%%%%%%%%%%%%%%%%%

\begin{keyword}
\kwd{information extraction}
\kwd{text mining}
\kwd{natural language processing}
\kwd{annotation}
\end{keyword}

% MSC classifications codes, if any
%\begin{keyword}[class=AMS]
%\kwd[Primary ]{}
%\kwd{}
%\kwd[; secondary ]{}
%\end{keyword}

\end{abstractbox}
%
%\end{fmbox}% uncomment this for twcolumn layout

\end{frontmatter}


%%%%%%%%%%%%%%%%%%%%%%%%%%%%%%%%%%%%%%%%%%%%%%
%% Main stuff
%%%%%%%%%%%%%%%%%%%%%%%%%%%%%%%%%%%%%%%%%%%%%%
\section{Introduction}

\todo[inline]{introduction, domain, long term goal, context of paper, corpus}

\subsection{Extraction task}

\todo[inline]{We are extracting on a graph based fashion, in order to build structures}

\todo{How is the level of detail here regarding the annotation scheme?}

Of interest in the domain is the extraction of changes to quantitative variables, such as \emph{increase in atmospheric CO$_2$}, and interactions between such changes, which can be either causal or correlative relations. Table \ref{ann_scheme} presents the trigger categories that were used to annotate text spans of interest in the corpus.

\begin{figure}
\Tree[.CATEGORY [.ENTITY \textit{Variable} \textit{Thing} ]
          [.EVENT [.\textit{Change} \textit{Increase} \textit{Decrease} ] 
         		  [.INTERACTION \textit{Cause} \textit{Correlate} ] 
          ] 
     ]
\caption{Type Hierarchy for the Annotation Scheme}
\label{ann_scheme}
\end{figure}

\emph{Variable} in our annotation scheme is defined as a \emph{quantitative variable}, meaning an entity than can be measured and assigned some value along some ordered axis, either as a numerical value or a value from a totally ordered set of discrete states. This includes among other counts, frequencies and ratios. Text spans that can be interpreted as quantitative variables out of context are abundant in the scientific literature, but only a small subset of these are of interest. Only variables that occur in an event are therefore annotated in the corpus. 

\emph{Thing} is used to annotate any span of text that functions as the argument of a change event, while not fulfilling the requirements to be annotated as a variable. 

A change event describes a directional, quantitative change in the value of a quantitative variable. \emph{Increase} is used to annotate a change in the positive direction, \emph{decrease} annotates changes in the negative direction and \emph{change} is used when the direction is underspecified in the text. All the change events take an entity as the \emph{theme} argument, signalling the variable that undergoes the change.

Interaction categories mark text spans that explicitly signal interactions between two change events. The types of interactions that are used in the corpus are \emph{cause} and \emph{correlate}. Cause events take two arguments: An \emph{agent}, which marks the change event that causes the other change event to come about. The \emph{theme} argument is used for the caused change event. Correlation events also take two arguments, the \emph{theme} and \emph{co-theme}. It was noticed during corpus annotation that there is a tendency in natural language to focus on one of the changes as initiating the other, giving such correlations a directional interpretation. For correlations with an directional interpretation, the initiating event is marked as the theme, and the other event is marked as the co-theme. If the correlation has no directional interpretation, the event that is the syntactic object of the correlation trigger is marked as co-theme, and the other event as theme by default.

Three additional categories are used in the corpus to handle some common language constructs, namely conjunction, disjunction and referring expressions. The categories \emph{and} and \emph{or} handle conjunctions and disjunctions, respectively, and each take two \emph{part} arguments. Referring expressions are handled by the \emph{RefExp} categories, which takes the antecedent as the \emph{coref} argument.

For a more detailed introduction to the annotation scheme used in the corpus, see \citet{mar14}.

\subsection{Event extraction approaches}

Event extraction is traditionally conducted with a bottom-up approach, in which entities are first extracted, and subsequently used as evidence during a second step in which events are extracted. However, it is also possible to conceive a top-down approach, where events are extracted first, and the second step extracts entities. 

The bottom-up approach is likely common in information extraction because in most extraction tasks, entity extraction is a separate, well-defined task that is often considered easier than event extraction. However, in the extraction task adapted here, it is impossible to clearly separate entity and event extraction, as text spans are only considered entities if they occur as the argument of an event.

It is therefore of interest to see whether the top-down approach is better suited for the task at hand than the bottom-up approach.
\section{Text mining in the climate change domain}

Research progress in the field of climate change is hampered by the wide range of relevant disciplines, which include, among others, climate science, earth science, oceanography, ecology and biochemistry. While a researcher may be able to keep up with the state-of-the-art within his/her field of specialization, it is impossible to keep track of all potentially useful findings in the related disciplines. Text mining can in this regards support the research effort by extracting structured and indexed representations of the findings in all relevant disciplines. This paper represents a line of work towards an information extraction systems to support scientific research in the domain of climate change science.

Climate change science aims to understand the causes and effects of climate change. The knowledge in the field can to a large extent be modelled as causal or correlative relations between variables, for instance \emph{increase in atmospheric CO$_2$ causes decrease in oceanic p.H.} or \emph{decrease in mineral nutrients correlates with decrease in phytoplankon abundance}. Our hypothesis is that this representional scheme is generic enough to represent fact accross all the relevant disciplines, yet specific enough to be useful to researchers in the domain. One observation that support this hypothesis is that this representational scheme is expressive enough to support inference of more complex structures, such as feedback loops \textemdash situations in which a change to a variable sets in motion a chain of events that, in turn, cause a new change to the original variable. Feedback loops are of particular interest to researchers in the field, due to the serious consequences of a potential feedback loop. 

This paper deals with automatic annotation rather than full information extraction. Annotation is the process of marking relevant text spans with category labels, and determining which relations hold between the marked text spans. For each task, the annotation guidelines specifies the set of category labels and relations which are used, and how they are used. The goal of automatic annotation is to automatically annotate unseen text with human-like annotation quality. To bridge the gap between automatic annotation and full information extraction, a post-processing procedure extracts the annotations and stores them as structured data. 

The annotation scheme used in the climate change science domain is described in detail in \cite{}. The remainder of this section will present the most important aspects of the annotation scheme. As an entry point, consider the following text fragment, taken from \cite{wal13} (some parentheses removed for clarity):

\begin{quote}

Increasing concentrations of CO2 cause a strong decline in growth, which decreases by up to 53\% over the investigated CO2 range.
Although the total carbon quota (TPC) is not affected by CO2, the organic carbon quota (POC) gradually increases while the inorganic carbon quota (PIC) shows a substantial decrease.

\end{quote}


The desired annotation for the sentences under the annotation scheme is presented in Figure [TODO: Erwin, I need your help for this]. The general annotation framework consists of two primitives: \emph{Triggers} and \emph{arguments}. Triggers are used to mark text spans that express interesting information of a specific type. Each trigger belongs to a category, taken from the categories presented in Table \ref{ann_scheme}. The usage of each category will be explained below. Arguments are directed links that mark relations between trigger spans. Arguments are typed, and certain types of trigger categories are required to take certain argument types, as explained below.

\begin{figure}
\Tree[.CATEGORY [.ENTITY \textit{Variable} \textit{Thing} ]
          [.EVENT [.\textit{Change} \textit{Increase} \textit{Decrease} ] 
         		  [.INTERACTION \textit{Cause} \textit{Correlate} ]
          ] 
     ]
\caption{Type Hierarchy of triggers in the Annotation Scheme}
\label{ann_scheme}
\end{figure}

\emph{Variable} is in our annotation scheme defined as a \emph{quantitative variable}, meaning an entity than can be measured and assigned some value along some ordered axis, either as a numerical value or a value from a totally ordered set of discrete states. This includes among other counts, frequencies and ratios. Examples of variables from the corpus include $CO_2$, \emph{organic carbon quota}, \emph{surface ocean pH} and \emph{mean light intensities}. Scientific articles normally contain a wide range of variables, but our annotation scheme limits itself to quantitative variables that occur in an event of interest, as only these can be used for making inferences by the downstream components. 

\emph{Thing} is used to annotate any span of text that functions as the argument of an event, but does not fulfil the requirements for being annotated as a variable, as described above. This occurs in phrases such as ''\emph{down-regulation of the gene}'', where \emph{down-regulation} clearly signals a quantitative change, and the only possible explicit argument of the change is \emph{the gene}. However, \emph{the gene} is not a variable. The variable that changes is the implicit \emph{activity level of the gene}, under the view that \emph{down-regulate} can be paraphrased as ''\emph{decrease the activity level of}''. The explicit argument is therefore annotated as a \emph{thing} rather than a \emph{variable}.

The underlying motivation for maintaining \emph{thing} and \emph{variable} as separate categories, is that the category \emph{thing} signals to the post-processing procedure that additional semantic interpretation should be factored in from the event trigger in order to specify the variable in question, as in the \emph{down-regulate} example. 

The change event categories are used to mark text spans that describe a directional, quantitative change in the value of a quantiative variable. \emph{Increase} is used to annotate a change in the positive direction, \emph{decrease} annotates changes in the negative direction and \emph{change} is used when the direction of the change is underspecified in the text. The change event takes the entity that undergoes the change as a \emph{theme} argument. A change event is not annotated unless an explicit entity that undergoes the signalled change can be inferred from the text.

Interaction categories mark text spans that explicitly signal interactions between two change events. The types of interactions that are used in the corpus are \emph{cause} and \emph{correlate}. Cause events take two arguments: An \emph{agent}, which marks the change event that causes the other change event to come about. The \emph{theme} argument is used for the caused change event. Correlation events also take two arguments, the \emph{theme} and \emph{co-theme}. It was noticed during corpus annotation that there is a tendency in natural language to focus on one of the changes as initiating the other, giving such correlations a directional interpretation. For correlations with an directional interpretation, the initiating event is marked as the theme, and the other event is marked as the co-theme. If the correlation has no directional interpretation, the event that is the syntactic object of the correlation trigger is marked as co-theme, and the other event as theme by default.

Sometimes an interaction holds between a change event and an entity or between two entities, rather than between two change events. An example of this is shown in the annotation of ''\emph{Although the total carbon quota (TPC) is not affected by CO2 \dots}''. Such entities are interpreted as the arguments of an implicit \emph{change}. 

It is not uncommon that the trigger word for a change event is used causatively, as in the sentence ''\emph{Heightened atmospheric CO$_2$ levels increase global temperatures}'', where \emph{increase} both signals an increase in the variable \emph{global temperatures} and signals a causal relationship between ''\emph{heightened atmospheric CO$_2$ levels}'' and ''\emph{increased global temperatures}''. In such cases, both change events are annotated normally, but in addition, the causative change event takes the other change event as an \emph{agent} argument, as if it were an interaction event.

One aspect in which our annotation scheme differs from the types of annotation schemes  traditionally used in biomedical corpora, such as GENIA\citep{kim08}, is that biomedical corpora normally annotate entities with types from a rich domain ontology, whereas our annotation scheme only subtypes entities into the broad categories \emph{variable} and \emph{thing}. The main reason for this is the lack of onotologies that cover all the disciplines related to climate change science.

\section{Methods}

An evaluation was conducted to compare the performance of the two information extraction approaches on the specified task. Statistical information extraction systems traditionally follow the bottom-up approach, and an existing information exaction system was therefore selected to provide the benchmark for the bottom-up approach. On the other hand, the top-down approach is rarely taken in the literature. It was therefore necessary to develop a prototype top-down information extraction system for the sake of comparison.

\subsection{Bottom-up system: TEES}

To act as a benchmark bottom-up system, the Turku Event Extraction System (TEES) \citep{bjö11ddi} was selected. TEES was selected for the evaluation because it is has shown state-of-the-art performance over several years of development, can extract events with arbitrary structures, can be trained with custom data sets and the source code is publicly available\footnote{\url{https://github.com/jbjorne/TEES}}. TEES has been developed for BioNLP, and has attained high scores on a number of shared tasks, including BioNLP 2009 (1st place) \citep{bjö09}, BioNLP 2011 (1st place in 4/8 tasks) \citep{bjö11} and Drug-Drug Interactions 2011 Challenge (4th place) \citep{bjö11ddi}.

The TEES pipeline consists of the following steps: Preprocessing, Named Entity Recognition, Trigger Detection, Edge Detection and Unmerging. 

Preprocessing consists of sentence splitting, constituency parsing and conversion to Stanford dependency representation. Sentence splitting is normally conducted using the GENIA sentence splitter\footnote{\url{http://www.nactem.ac.uk/y-matsu/geniass/}}, and parsing normally using the BLLIP-parser \citep{cha05} with the McClosky model for biomedical text \citep{mcc08}. Conversion to dependency format is normally performed by the Stanford dependency converter \citep{dem08}. Because the textual materials in the experiment conducted here were not of the biomedical domain, the Stanford parser \citep{kle03} was used instead for preprocessing.

The Named Entity Recognition, Trigger Detection, Edge Detection and Unmerging components are all machine learning-based multi-class classification components that use the $SVM^{MULTICLASS}$ implementation of SVM for classification.

TEES has participated mostly in event extraction-focused shared tasks, in which named entities were given as input to the system, so the Named Entity Component is normally not used, but can be included into the pipeline if required by the task, which is the case in the experiment presented here. The Named Entity Component makes a linear pass over all the tokens in the sentence, classifying every token as belonging to either one of the entity categories or as not an entity. Normally each single token is counted as an independent entity, but if two or more subsequent tokens are classified as belonging to the same class, they are merged to form a multi-token entity if a multi-token entity with the exact same text string can be found in the training data.

Trigger Detection makes another linear pass over the tokens in the sentence, detecting event trigger words. Every token is either classified as belonging to one of the event categories, or as not an event trigger. In the same way as with NER, multiple subsequent tokens can be merged to a single trigger if given the same class, and a trigger with the exact same text string can be found in the training data.

Edge Detection detects arguments of the triggered events. Every event is paired with every other event and entity, and the classifier determines which argument relation type holds between the pair, if any. Because this scheme lets events take not only entities, but also events as argument, this can create events with any structure desired.

In case of overlapping events, i.e. events that share some trigger, some cleaning up is required to separate the events. This is performed by the Unmerging step, which uses a classifier to determine whether an argument branch should be unmerged into a separate event. 

\subsection{Top-down system: Pattern matching}

In order to benchmark the top-down approach, an information extraction system had to be developed that followed this approach. Inspection of the data revealed that trigger words for events were relatively unambiguous. It was therefore concluded that a deterministic pattern matching system would be sufficient to provide an estimate for the performance of the top-down approach on the task at hand.

In the system, a pattern is defined as a partial dependency graph that signals the presence of an event. During information extraction, the system parses the text with the Stanford parser \citep{kle03}, and an event is detected for every pattern that matches the dependency parse tree of the sentence. In case of a match, each pattern specifies which node(s) in the dependency tree should be labelled as the event trigger, and which parts of the dependency tree that should be labelled as the arguments. 

A set of patterns was manually developed from eight of the papers in the corpus, with the remaining two held out as test data. For every annotated event in the corpus, a pattern was written that would extract that event. To improve performance, each pattern was manually evaluated against the eight development papers, and patterns that yielded a high ratio of false positives on the development data were excluded.

As an illustrative example, Figure \ref{dep_pattern} presents how the pattern \emph{T prep "in" pobj S} with $T=increase$ is matched against the sentences "This reveals a significant increase in jelly depositions." The successful match produces the annotation "This reveals a significant [increase$_{increase}$] in [jelly depositions$_{variable}$]."

\begin{figure}
\begin{center}
\begin{dependency}[theme = simple]
	\begin{deptext}
	This \& reveals \& a \& significant \& \textcolor{blue}{increase} \& \textcolor{red}{in} \& \textcolor{brown}{jelly} \& \textcolor{brown}{depositions} \& . \\
	\end{deptext}
	\depedge{2}{1}{NSUBJ}
	\depedge[arc angle=60]{5}{3}{DET}
	\depedge[arc angle=25]{5}{4}{AMOD}
	\depedge{2}{5}{DOBJ}
	\depedge[edge style=red]{5}{6}{PREP}
	\depedge[arc angle=25, edge style=brown]{8}{7}{NN}
	\depedge[edge style=red]{6}{8}{POBJ}
\end{dependency}
\end{center}
\caption{Matching against the dependency parse tree.}
\label{dep_pattern}
\end{figure}

\subsection{Why do we do it this way?}

\todo[inline]{I need some help regarding where to place this stuff and what to write}


\subsection{Evaluation methods}

TEES was evaluated by five-fold cross-validation, each fold consisting of two papers. As eight papers had been used during development of the pattern matching system, the pattern matching system was evaluated once only, using the two remaining papers as test data.

In the evaluation, an element is matched to an element in the gold standard if the text spans exhibit any degree of overlap, thus abstracting away from exact scoping of text spans. A predicted trigger was counted as a true positive if it matched an element of the same annotation category in the gold standard, and a false positive if it didn't. A trigger in the gold standard was counted as a false negative if it could not be matched to a predicted trigger of the same category. 
\section{Experiments}

This section presents the experiments. 

\subsection{Test data}

The annotated corpus consists of 10 full length scientific papers that were selected by a domain expert as represenative for the domain. The methods and materials section of the papers was not annotated, because this section consistently failed to yield any relevant information. All papers were taken from the open access journal PLOS ONE\footnote{\url{www.plosone.org}}, in order to avoid copyright issues when sharing the annotated corpus and any extracted material. 

The annotation itself was conducted using the brat rapid annotation tool\ref{ste12}. All 10 papers were annotated by author EAA, and subsequently reviewed by author EM. All disagreements were solved through discussion. Additional papers have been added to the corpus since the conclusion of the experiment presented here. The corpus is publicly available in its entirety through GitHub\footnote{\url{https://github.com/OC-NTNU/OCC}}.

\subsection{Evaluation methods}

TEES was evaluated by five-fold cross-validation, each fold consisting of two papers. As eight papers had been used during development of the pattern matching system, the pattern matching system was evaluated once only, using the two remaining papers as test data.

In the evaluation, an element is matched to an element in the gold standard if the text spans exhibit any degree of overlap, thus abstracting away from exact scoping of text spans. This was done to not penalize TEES for its biomedical-centric approach to extracting multiword expressions. A predicted trigger was counted as a true positive if it matched an element of the same annotation category in the gold standard, and a false positive if it didn't. A trigger in the gold standard was counted as a false negative if it could not be matched to a predicted trigger of the same category. 

\section{Results}

Precision, recall and f-score for each category for both systems were calculated and presented in Tables \ref{trigger_ev} and Table \ref{argument_ev}. 

\begin{table}
\begin{center}
\begin{tabular}{ | l | l | l | l | l | l | l | }
	\hline
	\cellcolor{gray} & \multicolumn{3}{c}{TEES} & \multicolumn{3}{c|}{PMS} \\ \hline
	\cellcolor{gray} & Precision & Recall & F-score & Precision & Recall & F-Score \\ \hline
	Variable & 0.41 & 0.40 & 0.40 & 0.71 & 0.51 & 0.59 \\
	Thing & 0.33 & 0.14 & 0.20 & 0.5 & 0.22 & 0.31 \\
	Increase & 0.74 & 0.46 & 0.57 & 0.97 & 0.78 & 0.86 \\
	Decrease & 0.76 & 0.39 & 0.52 & 0.88 & 0.71 & 0.79 \\ 
	Change & 0.47 & 0.16 & 0.24 & 0.72 & 0.64 & 0.68 \\ 
	Cause & 0.11 & 0.01 & 0.02 & 0.72 & 0.33 & 0.45 \\ 
	Correlate & 0.68 & 0.13 & 0.22 & 1.00 & 0.01 & 0.02 \\ \hline
	Macro & 0.50 & 0.24 & 0.31 & 0.79 & 0.46 & 0.53 \\
	Micro & 0.50 & 0.33 & 0.38 & 0.81 & 0.51 & 0.58 \\ \hline
\end{tabular}
\end{center}
\caption{Evaluation metrics, trigger categories.}
\label{trigger_ev}
\end{table}

\begin{table}
\begin{center}
\begin{tabular}{ | l | l | l | l | l | l | l | }	
	\hline
	\cellcolor{gray} & \multicolumn{3}{c}{TEES} & \multicolumn{3}{c|}{PMS} \\ \hline
	\cellcolor{gray} & Precision & Recall & F-score & Precision & Recall & F-Score \\ \hline
	Agent & 0.26 & 0.02 & 0.04 & 0.22 & 0.06 & 0.09 \\
	Theme & 0.39 & 0.19 & 0.23 & 0.43 & 0.32 & 0.37 \\
	Co-theme & 0.32 & 0.06 & 0.10 & 0.00 & 0.00 & 0.00 \\ \hline
	Macro & 0.32 & 0.09 & 0.12 & 0.22 & 0.13 & 0.15 \\ 
	Micro & 0.36 & 0.14 & 0.18 & 0.36 & 0.26 & 0.30 \\ \hline
\end{tabular}
\end{center}
\caption{Evaluation metrics, argument categories.}
\label{argument_ev}
\end{table}

It should be kept in mind that particularly the results produced provide a low estimate of the performance of a mature information extraction system on the task. The pattern matching system used to provide the benchmark estimates is extremely primitive, and has a large potential for improvement, whereas TEES, being a machine learning based system, likely suffers from lack of training data.

It can be seen from Table \ref{trigger_ev} that the PMS outperforms TEES on the entity categories (\emph{variable}, \emph{thing}). This seems to confirm that the additional information provided by having extracted change events yields benefits that far outweight the error propagted from the earlier step.

For both systems, performance was higher on the change event categories (\emph{increase}, \emph{decrease} and \emph{change}) than on the entity categories, or in fact, on any other trigger cateogry group. This indicates that the change event categories is the group of categories that is easiest to recognize correctly in this extraction task. The PMS outperformed TEES on all change event categories. It is likely tha the performance of TEES is affected negatively by errors in the earlier step, as the features used for event trigger detection include the number of entities in the sentence and entities within a linear window of the word in question.

For the interaction events, the PMS clearly outperforms the TEES system on the \emph{cause} category. However, the opposite holds true for the \emph{correlation} category. This is most likely due to the fact that the most productive patterns for correlations were removed to avoid false positives.

Analysis of performance on argument categories does not bring any new insights: Performance on the \emph{theme} category is higher for the PMS, due to the increased number of correct change events detected. The PMS scores 0.00 for the \emph{co-theme} category, due to fact that the system at the current stage of development is unable to properly distinguish themes and co-themes in correlations. It is surprising that the performance for the \emph{agent} category is rather similar for the two systems, especially given that the PMS by far outperforms TEES on the cause category. A significant reason for this is likely the fact that the PMS cannot handle causative change events at current stage of development.

Overall, the PMS system generally outperforms TTES system. The results are not perfect due to data limitation for TEES and system primitivity for the PMS, and due to the fact that the two systems take different approaches to information extraction, but the results still show that there could be some merit to altering the traditional information extraction pipeline.


\section{Discussion and conclusion}

It is likely that much of the success of the alternative pipeline is due to the fact that it started by extracting from change events, which appear to be the easiest categories. Evidence found during this step facilitates extraction of the more difficult categories. This can be contrasted with the traditional pipeline system, which started from the quite difficult entity categories, and the error-prone evidence found the that step could not improve performance on the already quite easy events categories. This result may possibly be generalized to the hypothesis that a pipeline information extraction system will attain the best results by starting extraction with the easiest category group.

The hypothesis is supported by the following probability theoretic argument: Information extraction can be formulated finding $argmax_{\boldsymbol{X}}\ P(X_0, X_1, ..., X_n | Y)$, where $X_0 ... X_n$ are the extraction item classes (entities, events and relations). Because it is ofen infeasable to finding argmax over the joint probablity distibution, the pipeline approach approximates the global optimum by decomposing the joint distribution using the chain rule of probability $P(X_0, X_1 ... X_n | Y) = P(X_0 | Y)P(X_1 | X_0, Y)P(X_2 | X_1, X_0, Y) \dots P(X_n | X_{n-1}, ..., X_0, Y)$ and greedily optimizing each step in the chain, thus computing $\hat{X_0} = argmax_{X_0}\ P(X_0 | Y)$, $\hat{X_1} = argmax_{X_1}\ P(X_1 | \hat{X_0}, Y)$ ... . In this setup, any error in $X_0$ will aversely affect all subsequent steps, as they then will try to optimize based on a suboptimal variable assignment. On the other hand, an error in $X_n$ will only affect the step itself. To minimize the approximation error, one should put the item classes that are least error-prone and least dependent on information from the other item classes first.

Verifying the hypothesis empirically, and exploring other heuristics for determining the optimal pipeline architecture for an extraction task remain tasks for future research. The development of an information extraction system that can lets the user make arbitrary adjustments to the pipeline would make this line of research immediately applicable. A longer term goal is the development of a system that is able to automatically select the optimal pipeline based directly on the data.

Running the extaction experiments has also given insight into the strengths and weaknesses of our annotation scheme, and refining the annotation scheme to sort out unnecessary complexities also remains an important research task. For instance, experiments in reasoning have uncovered that it does not make sense to make a binary distinction between entities that require no semantic interpretation (\emph{variables}) and entities that require full semantic interpretation (\emph{things}), because entities always require semantic interpretation to a certain degree. As an example, consider \emph{growth} in the example sentence "Increasing concentrations of CO2 cause  a strong decline in growth...". \emph{Growth} is arguably a quantitative variable, but for this to be useful during reasoning, a semantic interpretation component must disambiguate further what thing in the real world that the growth pertains to. In future versions of the annotation scheme, the category \emph{thing} will therefore be deprecated, and the category \emph{variable} used in all cases. All entities will then be passed to the semantic interpretation module.

THERE IS PLENTY OF FUTURE WORK THAT WE CAN MENTION HERE, AMONG OTHERS ONTOLOGIES, ADDING CO-REF TREATMENT, VARIABLE DISAMBIGUATION, REASONING, LDB ETC. IS IT WORTH MENTIONING, OR JUST NOISE FOR THE READER? 




%%%%%%%%%%%%%%%%%%%%%%%%%%%%%%%%%%%%%%%%%%%%%%
%%                                          %%
%% Backmatter begins here                   %%
%%                                          %%
%%%%%%%%%%%%%%%%%%%%%%%%%%%%%%%%%%%%%%%%%%%%%%

\begin{backmatter}

\section*{Competing interests}
  The authors declare that they have no competing interests.

\section*{Author's contributions}
    	Text for this section \ldots

\section*{Acknowledgements}
  Much of the work presented in this article was conducted as a part of the Master's project of author Elias Aamot. Elias Aamot is currently affiliated with Google, Inc., but the work presented here was substantially conducted while affiliated with the Norwegian University of Science and Technology.
  
%%%%%%%%%%%%%%%%%%%%%%%%%%%%%%%%%%%%%%%%%%%%%%%%%%%%%%%%%%%%%
%%                  The Bibliography                       %%
%%                                                         %%
%%  Bmc_mathpys.bst  will be used to                       %%
%%  create a .BBL file for submission.                     %%
%%  After submission of the .TEX file,                     %%
%%  you will be prompted to submit your .BBL file.         %%
%%                                                         %%
%%                                                         %%
%%  Note that the displayed Bibliography will not          %%
%%  necessarily be rendered by Latex exactly as specified  %%
%%  in the online Instructions for Authors.                %%
%%                                                         %%
%%%%%%%%%%%%%%%%%%%%%%%%%%%%%%%%%%%%%%%%%%%%%%%%%%%%%%%%%%%%%

% if your bibliography is in bibtex format, use those commands:
\bibliographystyle{bmc-mathphys} % Style BST file
\bibliography{bib.bib}      % Bibliography file (usually '*.bib' )

\end{backmatter}
\end{document}
