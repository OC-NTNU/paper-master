\section{Discussion}

It should be kept in mind that particularly the results produced by the top-down system provide a low estimate of the performance of a complete information extraction system on the task, as the pattern matching system used to provide the benchmark estimates is extremely primitive, and has a large potential for improvement. The bottom-up system, TEES, being a machine learning based system, likely suffers from lack of training data, and the results produced by this system should therefore also be considered a low estimate of the performance of a bottom-up system on the task. These factors make the comparison less reliable than it could have been.

It can be seen from Table \ref{trigger_ev} that the top-down system outperforms bottom-up on the entity categories (\emph{variable}, \emph{thing}). This is somewhat different from what was expected, as one would expect error propagation to create more problems for the top-down system than for the bottom-up system. This might imply that the extraction task at hand does not lend itself easily to the bottom-up approach. 

The top-down system outperformed the bottom-up system on all change event categories (\emph{increase}, \emph{decrease} and \emph{change}). For both systems, performance was higher on change event categories than on entity categories, or in fact, on any other trigger category group. This indicates that the change event categories is group of categories that is easiest to recognize correctly in extraction task at hand.

For the interaction events, the top-down system clearly outperforms the bottom-up system on the \emph{cause} category.  However, the opposite holds true for the \emph{correlation} category. This is most likely due to the fact that the most productive patterns for correlations were removed to avoid false positives.

The grammatical categories (\emph{and}, \emph{or}) were not handled by the top-down pattern matching system, so no interesting comparison could be made for these categories.

Analysis of performance on argument categories does not bring any new insights: Performance on the \emph{theme} category is higher for the top-down system, due to the increase number of correct change events detected. The top-down system scores 0.00 for the \emph{co-theme} category, due to fact that the system at the current stage of development is unable to properly distinguish themes and co-themes in correlations, and the \emph{part} category is not detected, as the grammatical categories are not handled by the pattern matching system. It is surprising that the performance for the \emph{agent} category is rather similar for the two systems, especially given that the top-down system by far outperforms the bottom-up system on the cause category. A significant reason for this is likely the fact that the top-down system cannot handle causative change events at current stage of development.

Overall, the top-down system generally outperforms the bottom-up system. The results are less reliable than they could have been, due to factors such as data limitation and system maturity, as well as the fact that the systems take two different approaches to information extraction, but the results still show that there could be some merit to altering the traditional information extraction pipeline.

It is likely that much of the success of the top-down system is due to the fact that it started extraction from change events, which were shown to be the easiest categories, and used evidence found during this step to facilitate extraction of the more difficult categories. This can be contrasted with the bottom-up system, which started from the quite difficult entity categories, and the rather unreliable conclusions made during that step could not improve performance on the already easy change events categories. This result may possibly be generalized to the hypothesis that a pipeline information extraction system will attain the best results by starting extraction with the easiest category group. Investigating this hypothesis further remains a task for future research.