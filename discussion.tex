\section{Discussion}

It should be kept in mind that particularly the results produced by the top-down system provide a low estimate of the performance of a complete information extraction system on the task, as the pattern matching system used to provide the benchmark estimates is extremely primitive, and has a large potential for improvement. The bottom-up system used for the evaluation, TEES, has reached a mature step of development, and the results produced are therefore likely close to expected actual performance. However, the system has been developed for a different domain, and it is therefore likely that performance can be improved with a domain-tailored bottom-up system.

\todo[inline]{Maybe say something about data limitations and reliability of ML on the corpus?}

It can be seen from Table \ref{trigger_ev} that the top-down system outperforms bottom-up on the entity categories (Variable, Thing). As the bottom-up system starts with entity extraction, one would expect it to perform well on these categories, whereas in the top-down system, which treats entity extraction as a downstream task, one would expect errors from the event extraction step propagate to create problems for entity extraction. The results for these categories are therefore different for what would be expected, giving evidence that the extraction task at hand does not lend itself easily to the bottom-up approach. 

The top-down system outperformed the bottom-up system on all change event categories (Increase, Decrease and Change). For both systems, performance was higher on change event categories than on entity categories, or in fact, on any other trigger category group. This indicates that the change event categories are easier to recognize than the other categories in the annotation scheme used here.

For the interaction events, the top-down system clearly the bottom-up system on the \emph{cause} category, but, surprisingly, the opposite holds true for the \emph{correlation} category. Given the differences in performance on the change event categories, one would expect the top-down system to yield the best performance in both categories, but the top-down system yielded an extremely low recall on the correlation category. This is because, at the current state of development, technical issues prevented the development of high-coverage patterns for the correlation category.

The grammatical categories (And, Or) were not handled by the top-down pattern matching system, so no interesting comparison could be made for these categories.

Analysis of performance on argument categories does not bring any new insights: Performance on the \emph{theme} category is higher for the top-down system, due to the increase number of correct change events detected. The top-down system scores 0.00 for the \emph{co-theme} category, due to fact that the system at the current stage of development is unable to distinguish themes and co-themes in correlations, and the \emph{part} category is not detected, as the grammatical categories are not handled by the pattern matching system. It is surprising that the performance for the \emph{agent} category is rather similar for the two systems, especially given that the top-down system by far outperforms the bottom-up system on the cause category. However, not much effort had been put into the correct selection of arguments for the interaction categories during development of the pattern matching system, so it is likely that performance on this category will improve significantly in a more mature system.

Overall, it can be observed that the top-down system generally outperforms the bottom-up system, with some exceptions where the primitiveness of the prototype system drastically degrades performance. It is likely that much of the success of the top-down system is due to the fact that it started extraction from change events, which were shown to be the easiest categories, and used evidence found during this step to facilitate extraction of the more difficult categories. This can be contrasted with the bottom-up system, which started from the quite difficult entity categories, and the rather unreliable conclusions made during that step could not improve performance on the already easy change events categories.