\section{Introduction}

Information extraction is the task of extracting structured information from natural language text. Historically, information extraction research focused on named entity recognition (NER), with research effort slowly turning to more complex tasks such as relation or event extraction when state-of-the-art on NER had reached acceptable levels. 

When developing information extraction systems for a new domain, the development effort normally adheres to an incremental approach similar to historical development of the information extraction field, starting with named entity recognition, and subsequently working on relation or event extraction. The architecture of most state-of-the-art information extraction systems reflect this development pattern, as they are pipeline systems with first a NER component, followed by a relation/event extraction component, where the output of the first component is provided as input to the second component.

The drawback of a pipeline architecture is error propagation, as each component makes hard choices that cannot be undone by downstream components, even though new information may surface later in the extraction process. Some recent research effort has therefore focused on joint extraction architectures, where steps are conducted in parallel. However, joint extraction architectures have not reached the level of maturity required for state-of-the-art performance, and have other drawbacks, such as the increased complexity of learning a joint probability distribution, that make them infeasible for certain circumstances, such as when training data is scarce, and the lack of off-the-shelf joint information extraction systems. 

The traditional pipeline architecture for information extraction appears to have arisen not as a principled design decision, but rather accidentally by following the research effort. This paper therefore investigates another approach to the pipeline architecture, where events detection precedes entity detection, and shows that for certain extraction tasks, reversing the order of the pipeline can prove beneficial.  

\subsection{Extraction task}

Progress in research towards understanding the full range of causes and consequences of global warming is hampered by the wide range of relevant disciplines, which include, among others, climate science, earth science, oceanography, ecology and biochemistry. Text mining can be used to provide an overview of the research in the relevant disciplines, and help the researchers draw parallels or uncover hidden connections. As a first step towards a discovery support system in climate change science, an information extraction system is being developed.

To extract relations that are general enough to be useful across disciplines, but still specific enough to be useful for a researcher, the information extraction system targets the extraction of events where quantiative variables undergo a directional change, such as \emph{increase in atmospheric CO$_2$}, and interactions between such change events, which in the current system is restricted to causal and correlative relations. A pilot corpus consisting of 10 scientific journal articles has been annotated, with details described in \cite{mar14}. The remainder of this section provides a sufficient introduction to the annotation scheme as to understand the specifics of the experiment.

As an example, consider the following text fragment, taken from \cite{wal13} (some parentheses removed for clarity):

\begin{quote}

Increasing concentrations of CO2 cause a strong decline in growth, which decreases by up to 53\% over the investigated CO2 range.
Although the total carbon quota (TPC) is not affected by CO2, the organic carbon quota (POC) gradually increases while the inorganic carbon quota (PIC) shows a substantial decrease.

\end{quote}


Table \ref{ann_scheme} presents an overview of the trigger categories that are used to annotate text spans of interest. The desired annotation for the text fragment with these categories is presented in Figure [TODO]. The remainder of this sections explains the usage of these categories. 

\begin{figure}
\Tree[.CATEGORY [.ENTITY \textit{Variable} \textit{Thing} ]
          [.EVENT [.\textit{Change} \textit{Increase} \textit{Decrease} ] 
         		  [.INTERACTION \textit{Cause} \textit{Correlate} ] 
          ] 
     ]
\caption{Type Hierarchy for the Annotation Scheme}
\label{ann_scheme}
\end{figure}

\emph{Variable} in our annotation scheme is defined as a \emph{quantitative variable}, meaning an entity than can be measured and assigned some value along some ordered axis, either as a numerical value or a value from a totally ordered set of discrete states. This includes among other counts, frequencies and ratios. Scientific articles normally describe a wide range of variables, but our annotation scheme limits itself to quantitative variables that occur in an event, as only these can be used for making inferences by the downstream components. 

\emph{Thing} is used to annotate any span of text that functions as the argument of an event, but does not fulfil the the requirements for being annotated as a variable. This occurs in phrases such as ''\emph{down-regulation of the gene}'', where \emph{down-regulation} clearly signals a quantitative change, an the argument of the change is \emph{the gene}, but \emph{the gene} is not a variable. The variable that changes is rather the implicit \emph{activity level of the gene}, under the view that \emph{down-regulate} can be interpreted as ''\emph{decrease the activity level of}''. The argument is therefore annotated as a \emph{thing} rather than a \emph{variable}.

The underlying motivation for maintaining \emph{thing} and \emph{variable} as separate categories, is that the category \emph{thing} signalled to the downstream components that additional semantic interpretation should be factored in from the event trigger in order to disambiguate the variable in question, as in the \emph{down-regulate} example. However, experiments in reasoning have uncovered that it does not make sense to make a binary distinction between entities that require no semantic interpretation (\emph{variables}) and entities that require full semantic interpretation (\emph{things}), because entities require various degrees of semantic interpretation. As an example, consider \emph{growth} in the first example sentence. \emph{Growth} is arguably a quantitative variable, but to be useful during reasoning, a semantic interpretation component must disambiguate further what kind of growth this is. In the latest version of the annotation scheme, the category \emph{thing} has therefore been deprecated, and the category \emph{variable} used in all cases. All entities will then be passed to the semantic interpretation module.

A change event describes a directional, quantitative change in the value of a quantitative variable. \emph{Increase} is used to annotate a change in the positive direction, \emph{decrease} annotates changes in the negative direction and \emph{change} is used when the change direction is underspecified in the text. All the change events take an entity as the \emph{theme} argument, signalling the variable that undergoes the change. A change event cannot is not annotated unless there exists an entity that undergoes the signalled change.

Interaction categories mark text spans that explicitly signal interactions between two change events. The types of interactions that are used in the corpus are \emph{cause} and \emph{correlate}. Cause events take two arguments: An \emph{agent}, which marks the change event that causes the other change event to come about. The \emph{theme} argument is used for the caused change event. Correlation events also take two arguments, the \emph{theme} and \emph{co-theme}. It was noticed during corpus annotation that there is a tendency in natural language to focus on one of the changes as initiating the other, giving such correlations a directional interpretation. For correlations with an directional interpretation, the initiating event is marked as the theme, and the other event is marked as the co-theme. If the correlation has no directional interpretation, the event that is the syntactic object of the correlation trigger is marked as co-theme, and the other event as theme by default.

Sometimes an interaction holds between a change event and an entity or between two entities, rather than between two change events. An example of this is shown in the annotation of ''\emph{Although the total carbon quota (TPC) is not affected by CO2 \dots}''. Such entities are interpreted the arguments of an implicit \emph{change}. 

It is not uncommon that the trigger word for a change event is used causatively, as in the sentence ''\emph{Heightened atmospheric CO$_2$ levels increase global temperatures}'', where \emph{hurt} both signal an increase in the variable \emph{global temperatures} and signal a causal relationship between ''\emph{heightened atmospheric CO$_2$ levels}'' and ''\emph{increased global temperatures}''. In such cases, both change events are annotated normally, but in addition, the causative change event takes the other change event as an \emph{agent} argument, as if it were an interaction event.

Three additional categories are used in the corpus to handle some common language constructs, namely conjunction, disjunction and referring expressions. The categories \emph{and} and \emph{or} handle conjunctions and disjunctions, respectively, and each take two \emph{part} arguments. Referring expressions are handled by the \emph{RefExp} categories, which takes the antecedent as the \emph{coref} argument.

\subsection{Pipeline Architecture}

The canonical information extraction pipeline architecture uses a bottom-up approach, in which the lowest order categories (entities) are extracted first, and subsequently used as evidence during later steps when higher order categories (events) are extracted. However, it is not hard to imagine that such an approach will have problems when extracting structures as described in the annotation scheme above, as entities are extracted before events, but entities should only be extracted if they occur as an argument of an event. 

It is conceivable to use an approach that extracts categories in a top-down fashion, where the higher order categories are extracted first, and the lower order categories later. On one hand, a top-down approach fits the annotation scheme presented above better, as entities cannot be extracted unless they occur in an event. However, the converse also holds true, as events are not extracted unless they are have a complete argument structure, requiring the presence of entities. 

An experiment was conducted to compare the approaches for the extraction task at hand.
